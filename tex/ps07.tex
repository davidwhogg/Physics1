\documentclass[12pt]{article}
\usepackage{graphics}
\newcounter{problem}
\newcommand{\kg}{\mathrm{kg}}
\newcommand{\m}{\mathrm{m}}
\newcommand{\s}{\mathrm{s}}
\newcommand{\mps}{\m\,\s^{-1}}
\newcommand{\hp}{\mathrm{hp}}
\begin{document}
\thispagestyle{empty}

\section*{NYU Physics 1---Problem set 7}

Due Tuesday 2009 November 3 at the beginning of lecture.

\paragraph{Problem~\theproblem:}\refstepcounter{problem}%
A Japanese Shinkansen (bullet train) (of mass $M=10^6~\mathrm{kg}$),
moving at $v_S=300~\mathrm{km\,h^{-1}}$ with respect to the tracks,
hits and collides elastically with a superball (of mass
$m=30~\mathrm{g}$), which is initially at rest.  The front face of the
train is inclined at an angle of $\theta=45~\mathrm{deg}$ to the
horizontal, as shown here in the rest frame of the tracks.\\
\rule{0.1\textwidth}{0pt}
\resizebox{0.8\textwidth}{!}{\includegraphics{../mp/shinkansen.eps}}

\textsl{(a)}~Draw a diagram of the ball--train system, just before the
collision, in the center-of-mass rest frame.  Clearly show the
velocities of the ball and train in this frame.

\textsl{(b)}~Draw a diagram, just after the collision, in the
center-of-mass frame.  Clearly show the velocities.

\textsl{(c)}~Draw a diagram, just after, back in the rest frame of the
tracks.  What is the final speed and direction of the ball,
immediately after the collision, in the rest frame of the tracks?

\paragraph{Problem~\theproblem:}\refstepcounter{problem}%
What is the peak force between two pool balls in a pool shot?
Estimate the momentum transferred to the object ball in a hard pool
shot.  Estimate (or look up) masses and velocities.  For how long are
two pool balls in contact?  The time can be approximated by the length
of time it takes a sound wave to cross a pool ball.  Put it all
together and compare it to the force of gravity or the normal force
from the table on each ball.

\paragraph{Problem~\theproblem:}\refstepcounter{problem}%
\textsl{(a)}~Imagine an ideal car of mass $1000\,\kg$ with a
cross-sectional area $A=3\,\m^2$, drag coefficient $1$ (so the
air-resistance force is exactly $(1/2)\,\rho\,A\,v^2$) producing total
mechanical power of $P=130\,\hp$.  At time $t=0$, the car is traveling
at $5\,\mps$ in the $x$ direction.  Use a spreadsheet to compute the
velocity as a function of time if the driver puts ``petal to the
metal'' for the next $20\,\s$ (with a $0.1\,\s$ time-step) on a flat,
straight, $x$-direction road.  You will have to use the fact that the
force accelerating the car is the power over the velocity (why?) but
subtracting off the air resistance force.  Make a plot of the velocity
of the car as a function of time, with axes clearly labeled.  Note
that this is only a ``first-order'' integration: You only need to keep
track of the velocity, not the position.

\textsl{(b)}~Why did I not have you start at a velocity of $0\,\mps$?
What would limit the acceleration in the first fraction of a second in
this case---for a car starting from rest?  It isn't the power of the
engine!

\end{document}
