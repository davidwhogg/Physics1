\documentclass[12pt]{article}
\usepackage{graphics}
\begin{document}

\section*{NYU Physics 1---In-class Exam 2}

\vfill

\paragraph{Name:} ~

\paragraph{email:} ~

\paragraph{recitation:} ~

\vfill

This exam consists of two problems.  Write only in this booklet.  Be
sure to show your work.

\vfill ~

\clearpage

\section*{Problem 1}

In the diagram below, all surfaces are frictionless, and all strings
and pulleys are massless and frictionless.
\\
\resizebox{\textwidth}{!}{\includegraphics{../mp/pulley_ramp.eps}}

(a) Draw free body diagrams for both blocks, showing all forces
acting.

\vfill

(b) Find the accelerations of the two blocks, in terms of the labeled
quantities and anything else you need ($g$, for example).  Be sure to
show clearly the direction corresponding to positive acceleration for
each block.

\vfill ~

\clearpage

\section*{Problem 2}

The water pressure in a building is maintained by pumping water into a
water tower on the roof.  If the water tower is 100~m above the city
water mains (so the water has to be pumped a vertical 100~m), and the
people living in the building use water at an average rate of 20
liters per hour, what is the average mechanical power (energy per unit
time) output of the pump that keeps the water tower full?  Give your
answer in units of kilowatts (kW), where
$1~\mathrm{W}=1~\mathrm{J\,s^{-1}}$.  You will have to use the facts
that water has a density of $1~\mathrm{g\,cm^{-3}}$, a liter is
$1000~\mathrm{cm^3}$, and $g\approx 10~\mathrm{m\,s^{-2}}$.  Clearly
state any assumptions you need to make to get an answer.

\clearpage

[This page intentionally left blank for calculations or other work.]

\end{document}
