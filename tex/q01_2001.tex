\documentclass[12pt]{article}
\begin{document}

\section*{NYU Physics 1---In-class Exam 1}

\vfill

\paragraph{Name:} ~

\paragraph{email:} ~

\paragraph{recitation:} ~

\vfill

This exam consists of two problems.  Write only in this booklet.  Be
sure to show your work.

\vfill ~

\clearpage

\section*{Problem 1}

Imagine that you are standing on the west bank of a river, which flows
directly south at $3~\mathrm{m\,s^{-1}}$.  You want to swim to a point
on the east bank, directly east of you.  If you try to swim straight
across, you will get swept downstream; to travel directly east, you
have to swim at an angle.  You know that you can swim at
$5~\mathrm{m\,s^{-1}}$ in still water.

\noindent
(a)~Draw a vector diagram, showing the velocity of the river, and your
velocity with respect to the river.  The sum of these two velocities
is your net velocity.  \emph{Draw the diagram for the case in which
you swim at just the right angle to make your net velocity point
exacly eastward.}

\vfill

\noindent
(b)~What is the magnitude of your net velocity?  Recall that for a
right-angle triangle, $a^2+b^2=c^2$.

\vfill ~

\clearpage

\section*{Problem 2}

At time $t=0$, a stone is thrown vertically upward at
$5~\mathrm{m\,s^{-1}}$.  In a coordinate system with the $z$ direction
pointing vertically upward, sketch the position $z$, vertical velocity
$v_z$ and vertical acceleration $a_z$ of the stone as a function of
time, for the time inteval $0<t<1~\mathrm{s}$ on three graphs below.
Assume that the stone begins at $z=0$ at $t=0$.  Be as quantitative as
you can and mark relevant times, distances, speeds, and accelerations
on your graphs where possible.  If it makes your mathematics simpler,
feel free to use $g= 10~\mathrm{m\,s^{-2}}$.

\clearpage

[This page intentionally left blank for calculations or other work.]

\end{document}

% How many taxicabs are there in New York City?  Carefully state every
% assumption you make and every number you estimate on your way to
% getting an answer.  If you happen to \emph{know} the answer, show how
% you would estimate it if you did not happen to know it!
