\documentclass[12pt]{article}
%% BUGS:
%% - The \begin{problem} ... \end{problem} is only used in exams??

\usepackage{url, graphicx, epstopdf}

% page layout
\setlength{\topmargin}{-0.25in}
\setlength{\textheight}{9.5in}
\setlength{\headheight}{0in}
\setlength{\headsep}{0in}
\setlength{\parindent}{1.1\baselineskip}

% headers
\newcommand{\examheader}[1]{
\noindent
Name:\rule[-1ex]{0.60\textwidth}{0.1pt}
NetID:\rule[-1ex]{0.20\textwidth}{0.1pt}

\section*{{NYU Physics I} --- {#1}}
\setcounter{problem}{1}}

% problem formatting
\newcommand{\problemname}{Problem}
\newcounter{problem}
\newenvironment{problem}{%
  \addvspace{\baselineskip}\noindent\textbf{Problem~\theproblem:}\refstepcounter{problem}
}{%
  \par\addvspace{\baselineskip}
}

% words
\newcommand{\foreign}[1]{\textsl{#1}}
\newcommand{\vs}{\foreign{vs}}

% math
\newcommand{\dd}{\mathrm{d}}
\newcommand{\e}{\mathrm{e}}

% primary units
\newcommand{\rad}{\mathrm{rad}}
\newcommand{\kg}{\mathrm{kg}}
\newcommand{\m}{\mathrm{m}}
\newcommand{\s}{\mathrm{s}}

% secondary units
\renewcommand{\deg}{\mathrm{deg}}
\newcommand{\km}{\mathrm{km}}
\newcommand{\cm}{\mathrm{cm}}
\newcommand{\mm}{\mathrm{mm}}
\newcommand{\ft}{\mathrm{ft}}
\newcommand{\mi}{\mathrm{mi}}
\newcommand{\AU}{\mathrm{AU}}
\newcommand{\ns}{\mathrm{ns}}
\newcommand{\h}{\mathrm{h}}
\newcommand{\yr}{\mathrm{yr}}
\newcommand{\N}{\mathrm{N}}
\newcommand{\J}{\mathrm{J}}
\newcommand{\eV}{\mathrm{eV}}
\newcommand{\W}{\mathrm{W}}
\newcommand{\Pa}{\mathrm{Pa}}

% derived units
\newcommand{\mps}{\m\,\s^{-1}}
\newcommand{\mph}{\mi\,\h^{-1}}
\newcommand{\mpss}{\m\,\s^{-2}}

% random stuff
\sloppy\sloppypar\raggedbottom\frenchspacing\thispagestyle{empty}

\begin{document}

\section*{NYU Physics I---Problem Set 6}

Due Tuesday 2025 October 14 at 23:59 NYC time on Brightspace.
Don't forget to give credit to your sources and helpers, according to
the credit policy on the syllabus.

\paragraph{\problemname~\theproblem:}\refstepcounter{problem}%
In Lecture we dropped a pool ball from a height of about $1\,\m$. It
bounced off of the cement floor. Roughly what was the impulse
delivered to the ball by the floor? Make reasonable assumptions! And
remember that an impulse has a magnitude and a direction. And units!

\paragraph{\problemname~\theproblem:}\refstepcounter{problem}\label{elastic}%
Finish the elastic collision problem we didn't finish in Lecture on 2025-10-02:

\textsl{(a)}~Compute the momentum of each block, the kinetic energy of
each block, and the total momentum and kinetic energy in the lab
frame, before the collision.

\textsl{(b)}~Compute the center-of-mass velocity of the system by
dividing the total momentum by the total mass. Draw the system (that
is, label the blocks with their velocities and masses) in the
center-of-mass frame, before the collision.

\textsl{(c)}~Compute the momentum of each block, the kinetic energy of
each block, and the total momentum and kinetic energy in the
center-of-mass frame, before the collision. The total momentum should
be zero; if it isn't, then you have made a misake.

\textsl{(d)}~In the center-of-mass frame, in an elastic collision, the
only option is for the momenta after to have the same magnitudes as
the momenta before, but with different directions. Since we are
working in one dimension, the only non-trivial option is to make the
blocks bounce off of each other, and reverse their momenta. Reverse
them, and draw the system after the collision in the center-of-mass
frame.

\textsl{(e)}~Compute the momentum of each block, the kinetic energy of
each block, and the total momentum and kinetic energy in the
center-of-mass frame, after the collision. Do your total numbers equal those
in part \textsl{(c)} above? They should!

\textsl{(f)}~Now invert the transformation you made in going from part
\textsl{(a)} to part \textsl{(b)}; that is, transform \emph{back} to
the lab frame. Draw the system in the lab frame, after the collision.

\textsl{(g)}~Compute the momentum of each block, the kinetic energy of
each block, and the total momentum and kinetic energy in the lab
frame, after the collision. Do your total numbers equal those in part
\textsl{(a)} above? They should!

\paragraph{\problemname~\theproblem:}\refstepcounter{problem}%
Re-do \problemname~\ref{elastic}
but now for the left-hand block having
mass $M$ and the right-hand block having mass $m\ll M$; that is, solve
the extreme mass-ratio problem with the same initial velocities.
If you got a discrepancy in part \textsl{(g)}, explain why.

\paragraph{\problemname~\theproblem:}\refstepcounter{problem}\label{blocks}%
In Problem Set 3, Problem 3, you computed an acceleration $a$ for the
hanging block. Now consider the energy and work.

\textsl{(a)}~If the hanging block falls by a distance $h$, what is
the change in the potential energy of the hanging block, and how much
work is done by friction on the sliding block?

\textsl{(b)}~The work done by friction is \emph{lost} to heat, so if
the system is released from rest and slides by a distance $h$, the
kinetic energy of the system should rise to a value that is related to both the
potential energy difference and the heat lost. Get that relationship
right and compute the kinetic energy you expect the system to have
when the hanging mass has dropped by a distance $h$.

\textsl{(c)}~Now interpret your answer in terms of an acceleration.
That is, compute the constant acceleration $a$ that would make the
result you computed in part \textsl{(b)} work out right. You will have
to use that $h = (1/2)\,a\,t^2$ and $v=a\,t$, which are both relevant
for constant acceleration. Does your answer agree with what you got on
Problem Set 3?

\paragraph{\problemname~\theproblem:}\refstepcounter{problem}%
In the static problem below, a beam is held horizontal by a diagonal
string (cable or tether), and a sign hangs from that beam. The beam is
attached to the wall by a pivot that is effectively frictionless, and
the strings are (effectively) massless. What is the tension $T_1$ in
the upper string, and the force $\vec{F}$ (give $x$ and $y$
components) on the beam at the pivot?
\\ \includegraphics{../mp/hanging_sign.pdf}

\paragraph{Extra \problemname\ (will not be graded for credit):}%
In \problemname~\ref{blocks}, It was useful to think about
conservation of energy. Why \emph{wasn't} it useful to think about
conservation of momentum? What would we have had to take into account
to think of this problem in terms of momentum?

\end{document}
