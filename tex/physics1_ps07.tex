\documentclass[12pt]{article}
%% BUGS:
%% - The \begin{problem} ... \end{problem} is only used in exams??

\usepackage{url, graphicx, epstopdf}

% page layout
\setlength{\topmargin}{-0.25in}
\setlength{\textheight}{9.5in}
\setlength{\headheight}{0in}
\setlength{\headsep}{0in}
\setlength{\parindent}{1.1\baselineskip}

% headers
\newcommand{\examheader}[1]{
\noindent
Name:\rule[-1ex]{0.60\textwidth}{0.1pt}
NetID:\rule[-1ex]{0.20\textwidth}{0.1pt}

\section*{{NYU Physics I} --- {#1}}
\setcounter{problem}{1}}

% problem formatting
\newcommand{\problemname}{Problem}
\newcounter{problem}
\newenvironment{problem}{%
  \addvspace{\baselineskip}\noindent\textbf{Problem~\theproblem:}\refstepcounter{problem}
}{%
  \par\addvspace{\baselineskip}
}

% words
\newcommand{\foreign}[1]{\textsl{#1}}
\newcommand{\vs}{\foreign{vs}}

% math
\newcommand{\dd}{\mathrm{d}}
\newcommand{\e}{\mathrm{e}}

% primary units
\newcommand{\rad}{\mathrm{rad}}
\newcommand{\kg}{\mathrm{kg}}
\newcommand{\m}{\mathrm{m}}
\newcommand{\s}{\mathrm{s}}

% secondary units
\renewcommand{\deg}{\mathrm{deg}}
\newcommand{\km}{\mathrm{km}}
\newcommand{\cm}{\mathrm{cm}}
\newcommand{\mm}{\mathrm{mm}}
\newcommand{\ft}{\mathrm{ft}}
\newcommand{\mi}{\mathrm{mi}}
\newcommand{\AU}{\mathrm{AU}}
\newcommand{\ns}{\mathrm{ns}}
\newcommand{\h}{\mathrm{h}}
\newcommand{\yr}{\mathrm{yr}}
\newcommand{\N}{\mathrm{N}}
\newcommand{\J}{\mathrm{J}}
\newcommand{\eV}{\mathrm{eV}}
\newcommand{\W}{\mathrm{W}}
\newcommand{\Pa}{\mathrm{Pa}}

% derived units
\newcommand{\mps}{\m\,\s^{-1}}
\newcommand{\mph}{\mi\,\h^{-1}}
\newcommand{\mpss}{\m\,\s^{-2}}

% random stuff
\sloppy\sloppypar\raggedbottom\frenchspacing\thispagestyle{empty}

\begin{document}

\section*{NYU Physics I---Problem Set 7}

Due Tuesday 2025-10-21 by 23:59 NYC time on Brightspace.
Be sure to give credit to your sources and helpers.
Note that there are special rules (see the syllabus) for giving credit to LLM sources.
The staff of this course \emph{recommends against} making heavy reliance on LLMs if you want to learn the material well.

\paragraph{\problemname~\theproblem:}\refstepcounter{problem}%
What is the length of a classical pendulum (light string, heavy point
mass at the end) that oscillates in small oscillations with a
2-s period? use $g=9.8\,\mpss$, and give your answer to three digits of
accuracy. This is called a ``seconds pendulum'' (because it ticks off
seconds). Apparently it isn't a coincidence that the answer to this is
close to $1\,\m$!

\paragraph{\problemname~\theproblem:}\refstepcounter{problem}%
Imagine you have two springs of natural length $L$ and spring constant $k_1$.
You join them end-to-end to make a longer spring (of natural length $2\,L$).
Will this longer spring be stiffer than the single spring, or less stiff, or the same?
That is, what is the spring constant $k_2$ for the longer spring?
\textsl{Hint:} It might help to think about the free-body diagrams for both of the springs, in a situation in which they are being stretched by a force $F$.

\paragraph{\problemname~\theproblem:}\refstepcounter{problem}%
A typical adult is holding her or his left arm at a right angle, so the
upper arm is pointing straight down, and the forearm is pointing
horizontally forwards.  The hand is oriented palm-up.  The arm is holding a
$6\,\kg$ grocery bag by its handle in the hand.  Look up the
point of attachment of the relevant tendon and make sensible estimates
(or look them up) for all lengths and masses.  In what follows, treat
the ``hand plus forearm'' to be one monolithic object; that is, we
primarily want to understand the forces at or near the elbow.

\textsl{(a)} Draw a free-body diagram for the hand-plus-forearm
system, identifying all significant forces acting on it (including
from the bag handle, and don't forget the elbow joint---the contact
force from the upper arm bones).

\textsl{(b)} Compute the magnitudes and directions of all forces, and
the magnitudes and directions of all torques, taking the elbow to be
the axis of rotation (that is, the origin or reference point).  For
simplicity, take the tendon direction and joint contact force both to
be precisely vertical.  That is, treat all angles as being right
angles.  This is not a bad approximation.

\textsl{(c)} Look up the definition of ``mechanical advantage'' and
compute the mechanical advantage the grocery bag has over the tendon.
Why would evolution (such a brilliant designer) decide to put tendons
under this kind of stress?

\paragraph{\problemname~\theproblem:}\refstepcounter{problem}%
Consider a mass $M$ attached to a spring of natural (equilibrium)
length $\ell$, with spring constant $k$, and hanging from the
ceiling. The system is subject to gravity with gravitational
acceleration $g$.

\textsl{(a)}~Because gravity is acting, the equilibrium position of
the mass will not be at the equilibrium length of the string, but
instead stretched. Compute the amount the spring is stretched at
equilibrium.

\textsl{(b)}~Carefully choose the following coordinate system:
$\hat{y}$ points upwards (opposite to gravity), and the position $y=0$
is where the spring has length $\ell$. That is, the zero of the
coordinate system is where the spring would be unstretched in the
absence of gravity. Make $y=0$ also the zero of the gravitational
potential energy. In this coordinate system, plot the potential energy
as a function of $y$ position. You need to take into account both the
gravitational potential energy and the spring potential energy. Show
that the equilibrium position you computed in the previous part is the
minimum of this potential energy curve.

\textsl{(c)}~Compare the second derivative of the potential energy at
the minimum to the same for the same spring \emph{not} subject to a
gravitational force. What are the implications of this calculation?

\textsl{(d)}~Write down the differential equation relating $y$ to its
second derivative, and show that this differential equation is solved
by sinusoidal motion. What are the angular frequency and period of the
oscillating solutions?

\paragraph{\problemname~\theproblem:}\refstepcounter{problem}%
Determine the value of $\pi$ by integration!

\textsl{(a)}~Make a spreadsheet with a dimensionless column marked
``$t$'' that goes from 0.0 to 20.0 in steps of 0.05.  Now make columns
marked ``$c$'' and ``$s$''.  Integrate the functions $c(t)$ and
$s(t)$ with the properties that:
\begin{eqnarray}\displaystyle
  c(0) & = & 1.0 \\
  s(0) & = & 0.0 \\
  \Delta c & = & -s\,\Delta t \\
  \Delta s & = & c\,\Delta t
  \quad .
\end{eqnarray}
That is, you are integrating functions that are each other's derivatives:
\begin{eqnarray}\displaystyle
  \frac{\dd c(t)}{\dd t} &=& -s(t) \\
  \frac{\dd s(t)}{\dd t} &=& +c(t)
  \quad .
\end{eqnarray}
Make a graph of $c(t)$ vs $t$ and $s(t)$ vs $t$.  Remind you of anything?

\textsl{(b)}~Make an expanded graph in the region $1.5<t<1.6$ and
look where one of the curves crosses zero.  Multiply your answer by 2 and you
have an estimate of $\pi$!  Why does that work?

\textsl{(c)}~There are many estimates of $\pi$ you can make, given
your spreadsheet. What do you think will be the most accurate one and
why? Check.

\end{document}
