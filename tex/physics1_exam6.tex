\documentclass[12pt]{article} \usepackage{url, graphicx, epstopdf}

% page layout
\setlength{\topmargin}{-0.25in}
\setlength{\textheight}{9.5in}
\setlength{\headheight}{0in}
\setlength{\headsep}{0in}
\setlength{\parindent}{1.1\baselineskip}

% problem formatting
\newcommand{\problemname}{Problem}
\newcounter{problem}

% words
\newcommand{\foreign}[1]{\textsl{#1}}
\newcommand{\vs}{\foreign{vs}}

% math
\newcommand{\dd}{\mathrm{d}}
\newcommand{\e}{\mathrm{e}}

% primary units
\newcommand{\rad}{\mathrm{rad}}
\newcommand{\kg}{\mathrm{kg}}
\newcommand{\m}{\mathrm{m}}
\newcommand{\s}{\mathrm{s}}

% secondary units
\renewcommand{\deg}{\mathrm{deg}}
\newcommand{\km}{\mathrm{km}}
\newcommand{\cm}{\mathrm{cm}}
\newcommand{\mm}{\mathrm{mm}}
\newcommand{\ft}{\mathrm{ft}}
\newcommand{\mi}{\mathrm{mi}}
\newcommand{\AU}{\mathrm{AU}}
\newcommand{\ns}{\mathrm{ns}}
\newcommand{\h}{\mathrm{h}}
\newcommand{\yr}{\mathrm{yr}}
\newcommand{\N}{\mathrm{N}}
\newcommand{\J}{\mathrm{J}}
\newcommand{\eV}{\mathrm{eV}}
\newcommand{\W}{\mathrm{W}}
\newcommand{\Pa}{\mathrm{Pa}}

% derived units
\newcommand{\mps}{\m\,\s^{-1}}
\newcommand{\mph}{\mi\,\h^{-1}}
\newcommand{\mpss}{\m\,\s^{-2}}

% random stuff
\sloppy\sloppypar\raggedbottom\frenchspacing\thispagestyle{empty}

\begin{document}

\noindent
Name: \rule[-1ex]{0.55\textwidth}{0.1pt}
NetID: \rule[-1ex]{0.2\textwidth}{0.1pt}

\section*{NYU Physics I---Term Exam 6}

\paragraph{\problemname~\theproblem:}\refstepcounter{problem}%
What is the speed of a package orbiting on a circular orbit right near
the surface of the Earth? Give your answer in $\mps$.
(from Problem Set 11)

\vfill

\paragraph{\problemname~\theproblem:}\refstepcounter{problem}%
Sketch an orbit of roughly eccentricity 0.9. Most importantly: Show the
point about which the object is orbiting, and make sure your pericenter
and apocenter distances make sense. Don't worry about getting it all right,
just roughly!
(from Problem Set 12)

\vfill

\paragraph{\problemname~\theproblem:}\refstepcounter{problem}%
Draw a space-time diagram that shows a stationary base E (in the rest
frame of E) and a ship S moving at speed $0.5\,c$ in the $x$
direction with respect to E. At some time (your choice!) when the base and ship are far
apart, base E sends a light signal to the ship S. Draw that light
signal on your diagram too.
(from Problem Set 13)

\vfill
~
\clearpage

\paragraph{\problemname~\theproblem:}\refstepcounter{problem}%
Earth orbits on a nearly circular orbit at 1\,AU; Jupiter orbits
on a nearly circular orbit at 5.2\,AU. What is the semi-major axis
of the transfer orbit that just kisses both of these orbits?
(from lecture on 2018-11-27)

\vfill

\paragraph{\problemname~\theproblem:}\refstepcounter{problem}%
If you want to observe a time-dilated celebration (or, say, lifetime
of some unstable particles), time dilated by a factor of 10, how fast
does the party (or do the particles) have to move with respect to you?
Give your answer in terms of the speed of light $c$.
(from lecture on 2018-12-04)

\vfill

\paragraph{\problemname~\theproblem:}\refstepcounter{problem}%
What is the spacetime interval $(\Delta s)^2$ between the two events $A$ and $B$?
$$A = (c\,t_A, x_A) = (2\,\m, 6\,\m) $$
$$B = (c\,t_B, x_B) = (7\,\m, 3\,\m) $$
Don't forget your units. (from the recitation on the interval)

\vfill
~
\end{document}
