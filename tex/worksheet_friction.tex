\documentclass[12pt]{article}
\renewcommand{\deg}{\mathrm{deg}}
\newcommand{\kg}{\mathrm{kg}}
\newcommand{\m}{\mathrm{m}}
\newcommand{\s}{\mathrm{s}}
\newcommand{\mps}{\m\,\s^{-1}}
\newcounter{problem}
\begin{document}\sloppy\sloppypar\raggedbottom\frenchspacing\thispagestyle{empty}

\section*{NYU Physics 1---friction}

In lecture you did a problem of a block on an inclined plane, but for
the somewhat complicated case of a car sliding around a banked turn.
Here we consider the simpler ``block on plane'' problem but with
friction.  When we ``switch on'' friction, the contact force between
the block and the plane is no longer purely normal but has both normal
and transverse (frictional) components.  For definiteness, imagine a
plane or bank inclined at about $20\,\deg$ to the horizontal in what
follows.

\paragraph{\theproblem}\refstepcounter{problem}%
Work out the problem of a block on an inclined plane in the
\emph{absence of friction}.  That is, compute the magnitude of the
normal force and the magnitude and direction of the acceleration of
the block.

\paragraph{\theproblem}\refstepcounter{problem}%
Imagine that between the block and the plane there is a coefficient of
friction $\mu=0.05$.  Draw the forces on the block, separating the
contact force into its normal and transverse components.

\paragraph{\theproblem}\refstepcounter{problem}%
Solve for the frictional (transverse) component of the contact force.
Do you get a magnitude of $\mu\,m\,g\,\cos\theta$?  Which way does it
point?

\paragraph{\theproblem}\refstepcounter{problem}%
What is the acceleration of the block in this case?

\paragraph{\theproblem}\refstepcounter{problem}%
Describe a situation in which the frictional force would point
\emph{down} the plane.  When you answered the previous question, you
made an assumption.  What was it?

\paragraph{\theproblem}\refstepcounter{problem}%
Now imagine that $\mu=0.9$.  What is the magnitude of the frictional
force?  Explain why it \emph{cannot} be $\mu\,m\,g\,\cos\theta$.

\end{document}
