\documentclass[12pt]{article}
\newcounter{problem}
\begin{document}\thispagestyle{empty}

\section*{NYU General Physics 1---Problem set 9}

\paragraph{Problem~\theproblem:}\refstepcounter{problem}%
A very thin ladder of length $L$ and mass $M$ leans against a vertical
wall, on a horizontal floor, making an angle of $\theta$ with respect
to the wall.  Imagine that there is a large coefficient of friction
$\mu$ at the floor so that the ladder is in static
equilibrium, but assume that the wall is effectively frictionless.

\textsl{(a)} Draw a free-body diagram for the ladder, showing all
forces acting.

\textsl{(b)} Using the bottom of the ladder as the axis of rotation or
origin, compute all the forces and torques on the ladder such that it
is in equilibrium.

\textsl{(c)} Why did I make the wall ``effectively frictionless''?

\textsl{(d)} Re-solve the problem using the \emph{top} of the ladder
as the axis of rotation or origin.  What is different in the end?

\textsl{(e)} At what angles $\theta$ would the ladder start to slip?
If $\mu=0.8$ (not unreasonable for rubber ladder feet on a wood
floor), what is the maximum angle at which you could lean the ladder?

\paragraph{Problem~\theproblem:}\refstepcounter{problem}%
A long, thin rod of length $L$ and cross-sectional area $A$ and
elastic (Young's) modulus $E$ has mass $M$.

\textsl{(a)} Think of the rod as being like a Hooke's Law spring; it
can be stretched by applying a force.  What is the spring constant $k$
for this spring?

\textsl{(b)} By dimensional analysis, can you combine $L$, $A$, $E$,
and $M$ into a frequency?  Do you have more than one choice?  If so,
which of the choices makes most sense?

\textsl{(c)} Look up the properties of a femur bone and compute this
frequency for the femur bone.

\textsl{(d)} Repeat part \textsl{(c)} but replacing the mass $M$ of
the femur bone with the mass of a typical college-age human.  Hold
everything else constant.  That is, think of this problem as being a
human mass on a femur-bone spring.

\textsl{(e)} Compare the frequency you got in part \textsl{(c)} to the
dimensional analysis frequency you can obtain by combining the
acceleration due to gravity $g$ with $L$ and $M$.  What does this
frequency represent?  Is it higher or lower?  Does that jive with your
intuition?

\paragraph{Problem~\theproblem:}\refstepcounter{problem}%
Consider a mass $M$ on a spring of spring constant $k$, released from
rest but from a distance $X$ (in the $x$-direction, which is parallel
to the spring) away from the equilibrium position for the mass.
Subsequently, the mass on the spring oscillates without any loss of
energy.  Plot the position $x$ as a function of time, the velocity $v$
as a function of time, the acceleration $a$, the kinetic energy $K$,
the spring potential energy $U$, and the total energy $E$.  Make your
plots in a time-aligned stack so that you can compare them.

\end{document}
