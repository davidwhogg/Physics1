\documentclass[12pt]{article}
\usepackage{graphics}
\newcommand{\kg}{\mathrm{kg}}
\newcommand{\m}{\mathrm{m}}
\newcommand{\s}{\mathrm{s}}
\newcommand{\mps}{\m\,\s^{-1}}
\newcommand{\N}{\mathrm{N}}
\newcommand{\Npm}{\N\,\m^{-1}}
\newcommand{\Nspm}{\N\,\s\,\m^{-1}}
\newcommand{\hp}{\mathrm{hp}}
\newcommand{\percent}{\mathrm{percent}}
\newcommand{\mph}{\mathrm{mi}\,\mathrm{h}^{-1}}
\renewcommand{\vector}[1]{\mathbf{\vec{#1}}}
\newcounter{problem}
\begin{document}
\thispagestyle{empty}

\section*{NYU Physics 1---Problem set 6}

Due Tuesday 2009 October 27 at the beginning of lecture.

\paragraph{Problem~\theproblem:}\refstepcounter{problem}%
Determine the acceleration $\vector{a}$ of block $m_1$ and the
frictional force $\vector{f}$ acting on block $m_1$ in this system.
Use $m_1= 30\,\kg$, $m_2=10\,\kg$ and do the three cases of $\mu=0$,
$\mu=0.1$, and $\mu=1.0$.  Assume that the pulley and strings are
massless and frictionless.\\
~\hfill\includegraphics{../mp/pulley_table.eps}\hfill~

\paragraph{Problem~\theproblem:}\refstepcounter{problem}%
Imagine a mass $m$ on a spring moving in the $x$ direction, subject to
a $x$-direction restoring force $-k\,x$ and subject to a ``viscous
damping'' fource $-\beta\,v_x$ (that is, proportional to velocity but
acting opposite to the direction of the velocity).

\textsl{(a)}~Show that the differential equation for this system can
be written as
\begin{equation}
\frac{\mathrm{d^2}x}{\mathrm{d}t^2}
+\gamma\,\frac{\mathrm{d}x}{\mathrm{d}t}
+\omega^2\,x = 0 \quad ,
\end{equation}
where $\gamma$ and $\omega$ are constants.  What are the units of
$\gamma$ and $\omega$, and how are each of these constants related to
$k$, $\beta$, and $m$?

\textsl{(b)}~Make a spreadsheet with $t$, $x$, $v_x$, and $a_x$, with
time going from 0 to $10.0\,\s$ in units of $0.01\,\s$, $x$ starting
at $1.0\,\m$, and $v_x$ starting at $0.0\,\mps$, for a system with
$m=1.0\,\kg$, $\beta=1.0\,\Nspm$, and $k=100\,\Npm$.  The spreadsheet
should show an oscillation but with a decaying amplitude.

\textsl{(c)}~Make plots of the kinetic energy $K$, potential energy
$U$, and the sum $K+U$.  This energy ought to decrease with time.  By
what factor does $K+U$ decrease in $10\,\s$?  Where is the energy
going, physically?

\paragraph{Problem~\theproblem:}\refstepcounter{problem}%
A typical cheap American car has a mass of $1000\,\kg$, a
cross-sectional area of $3\,\m^2$, and produces $100\,\hp$ of
mechanical power.

\textsl{(a)}~If air resistance doesn't matter, what is the time for
the car to go from 0 to $60\,\mph$ if all this power really is used
for acceleration?  Ignore air resistance.

\textsl{(b)}~If air resistance \emph{does} matter, what is the top
speed at which this car can drive, in $\mph$.  That is, at what speed
does the $100\,\hp$ equal the power done against air resistance?  Use
the air resistance approximation $F=\rho\,A\,v^2$.

\textsl{(c)}~By what factor will the top speed be increased if the
power of the car is increased to $130\,\hp$?

\textsl{(d)}~Return to $100\,\hp$.  If all of that horsepower comes
from gasoline, and if that gasoline is used at $100\,\percent$
efficiency to make mechanical power (this assumption is off by a
factor of $2$), how many miles per gallon will the car get at its top
speed?  Use the energy content of gasoline you computed earlier, or
else look up a value on the web.  Be clear what you are using to
calculate.

\textsl{(e)}~By what factor will the miles per gallon be increased if
the car drives at half that speed?  All the assumptions are suspect,
but the message is clear!

\paragraph{Extra Problem (will not be graded for credit):}
Imagine that the air resistance faced by an automobile can be thought
of as tiny elastic collisions with $\mathrm{N}_2$ molecules, all
initially at rest.  Model the car as a rectangular parallelepiped
bashing through the air at speed $v$.  What do you get for the air
resistance force $F$ in this case?  Give your answer in terms of the
speed $v$, the cross-sectional area $A$ and the mass density of air
$\rho$.  Note that the average force on the car can be estimated by
the total mean impulse in a time interval, divided by that time
interval (momentum per time is force).  How do you think your answer
will change if you make the $\mathrm{N}_2$ molecules move in random
directions to start, instead of starting with them at rest?

\end{document}
