\documentclass[12pt]{article}
\usepackage{url, graphicx, epstopdf}

% page layout
\setlength{\topmargin}{-0.25in}
\setlength{\textheight}{9.5in}
\setlength{\headheight}{0in}
\setlength{\headsep}{0in}
\setlength{\parindent}{1.1\baselineskip}

% problem formatting
\newcommand{\problemname}{Problem}
\newcounter{problem}

% words
\newcommand{\foreign}[1]{\textsl{#1}}
\newcommand{\vs}{\foreign{vs}}

% math
\newcommand{\dd}{\mathrm{d}}
\newcommand{\e}{\mathrm{e}}

% primary units
\newcommand{\rad}{\mathrm{rad}}
\newcommand{\kg}{\mathrm{kg}}
\newcommand{\m}{\mathrm{m}}
\newcommand{\s}{\mathrm{s}}

% secondary units
\renewcommand{\deg}{\mathrm{deg}}
\newcommand{\km}{\mathrm{km}}
\newcommand{\cm}{\mathrm{cm}}
\newcommand{\mm}{\mathrm{mm}}
\newcommand{\ft}{\mathrm{ft}}
\newcommand{\mi}{\mathrm{mi}}
\newcommand{\AU}{\mathrm{AU}}
\newcommand{\ns}{\mathrm{ns}}
\newcommand{\h}{\mathrm{h}}
\newcommand{\yr}{\mathrm{yr}}
\newcommand{\N}{\mathrm{N}}
\newcommand{\J}{\mathrm{J}}
\newcommand{\eV}{\mathrm{eV}}
\newcommand{\W}{\mathrm{W}}
\newcommand{\Pa}{\mathrm{Pa}}

% derived units
\newcommand{\mps}{\m\,\s^{-1}}
\newcommand{\mph}{\mi\,\h^{-1}}
\newcommand{\mpss}{\m\,\s^{-2}}

% random stuff
\sloppy\sloppypar\raggedbottom\frenchspacing\thispagestyle{empty}

\begin{document}

\section*{NYU Physics I---Problem Set 11}

Due Tuesday (yes, Tuesday) 2016 November 29 at the beginning of lecture.

\paragraph{\problemname~\theproblem:}\refstepcounter{problem}%
Circular orbits in gravity

\textsl{(a)}~Draw a free-body diagram for a package orbiting the Earth
on a circular orbit at radius $a$, outside the atmosphere. Ignore all
gravitational forces other than that from Earth.

\textsl{(b)}~At what speed $v$ and orbital period $T$ must the package
move such that its orbit will be circular?

\textsl{(c)}~What is the orbital time at the surface of the Earth, and
at the altitude of the orbit of the International Space Station?

\textsl{(d)}~What is the orbital radius of a satellite in geosynchronous
orbit? Compute it, and then check it by looking it up on the internets.

\textsl{(e)}~Look up Kepler's laws and see if your answers to
\textsl{(c)} and \textsl{(d)} are consistent with all of them.

\paragraph{\problemname~\theproblem:}\refstepcounter{problem}%
bar

\paragraph{\problemname~\theproblem:}\refstepcounter{problem}%
Look up the value and units of Newton's constant $G$.

\textsl{(a)}~Find some combination of Newton's constant $G$, the speed of light $c$, and
some (arbitrary) mass $M$, that has units of \emph{length}.

\textsl{(b)}~Evaluate your expression for the mass of the Sun (you
will have to look it up); that is, find the characteristic length
associated with that mass.  Give your answer in units of km.  What is
the physical meaning of this length (approximately)?

\paragraph{Extra Problem (will not be graded for credit):}%
Compute the typical force on the Moon from the Earth, and then compute
the typical force on the Moon from the Sun. Why was it that you were
permitted to ignore the force on the Moon from the Sun in the first
problem, above? It isn't because the force on the Moon from the Sun is
small!

\end{document}
