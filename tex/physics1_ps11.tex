\documentclass[12pt]{article}
\usepackage{url, graphicx, epstopdf}

% page layout
\setlength{\topmargin}{-0.25in}
\setlength{\textheight}{9.5in}
\setlength{\headheight}{0in}
\setlength{\headsep}{0in}
\setlength{\parindent}{1.1\baselineskip}

% problem formatting
\newcommand{\problemname}{Problem}
\newcounter{problem}

% words
\newcommand{\foreign}[1]{\textsl{#1}}
\newcommand{\vs}{\foreign{vs}}

% math
\newcommand{\dd}{\mathrm{d}}
\newcommand{\e}{\mathrm{e}}

% primary units
\newcommand{\rad}{\mathrm{rad}}
\newcommand{\kg}{\mathrm{kg}}
\newcommand{\m}{\mathrm{m}}
\newcommand{\s}{\mathrm{s}}

% secondary units
\renewcommand{\deg}{\mathrm{deg}}
\newcommand{\km}{\mathrm{km}}
\newcommand{\cm}{\mathrm{cm}}
\newcommand{\mm}{\mathrm{mm}}
\newcommand{\ft}{\mathrm{ft}}
\newcommand{\mi}{\mathrm{mi}}
\newcommand{\AU}{\mathrm{AU}}
\newcommand{\ns}{\mathrm{ns}}
\newcommand{\h}{\mathrm{h}}
\newcommand{\yr}{\mathrm{yr}}
\newcommand{\N}{\mathrm{N}}
\newcommand{\J}{\mathrm{J}}
\newcommand{\eV}{\mathrm{eV}}
\newcommand{\W}{\mathrm{W}}
\newcommand{\Pa}{\mathrm{Pa}}

% derived units
\newcommand{\mps}{\m\,\s^{-1}}
\newcommand{\mph}{\mi\,\h^{-1}}
\newcommand{\mpss}{\m\,\s^{-2}}

% random stuff
\sloppy\sloppypar\raggedbottom\frenchspacing\thispagestyle{empty}

\begin{document}

\section*{NYU Physics I---Problem Set 11}

Due Tuesday (yes, Tuesday) 2018 November 27 at the beginning of lecture.

\paragraph{\problemname~\theproblem:}\refstepcounter{problem}%
\textsl{(a)}~Compute the spin angular momentum of the Earth in SI
units; that is, compute the angular momentum that comes from the fact
that the Earth rotates one rotation every 23 hours and 56 minutes (or
24 hours is good enough; why the difference?). Assume that the Earth
is a sphere for these purposes, and choose the center of the Earth as
your reference point (axis).

\textsl{(b)}~Compute the orbital angular momentum of the Earth in SI
units; that is, compute the angular momentum that comes from the fact
that the Earth goes around the Sun once every 365.256363004
days. (Give your answer to 2 digits of accuracy, not 12!) Assume that
the Earth is a point particle for these purposes, and choose the
center of the Sun as your reference point (axis). For both this
problem and the previous, you might have to look up radii and masses
on the internets.

\paragraph{\problemname~\theproblem:}\refstepcounter{problem}\label{package}%
\textsl{(a)}~Draw a free-body diagram for a package orbiting the Earth
on a circular orbit at radius $a$, outside the atmosphere. Ignore all
gravitational forces other than that from Earth.

\textsl{(b)}~At what speed $v$ and orbital period $T$ must the package
move such that its orbit will be circular?

\textsl{(c)}~What is the orbital time at the surface of the Earth, and
at the altitude of the orbit of the International Space Station?

\textsl{(d)}~What is the orbital radius of a satellite in geostationary
orbit? Compute it, and then check it by looking it up on the internets.

\textsl{(e)}~Look up Kepler's laws and see if your answers to
\textsl{(c)} and \textsl{(d)} are consistent with all of them.

\paragraph{\problemname~\theproblem:}\refstepcounter{problem}\label{rocket}%
\textsl{(a)} What is the total mechanical energy of a package of mass
$m= 1~\mathrm{kg}$ sitting on the surface of the Earth on the equator?
Take as the zero of potential energy a motionless package at infinity,
and don't forget to include the kinetic energy from the fact that the
package is sitting on the rotating Earth.  Give your answer both in
symbols (you will need to have symbols for the radius of the Earth,
mass of the Earth, and period of rotation of the Earth) and also in J.

(b) To get the package from sitting on the Earth into orbit most
easily, should you launch it to the north, south, east or west?
Explain your reasoning.

(c) What is the total mechanical energy of the package orbiting just
above the surface of the Earth? Again, symbols and also in J.

(d) What is the total energy of the package in geostationary
orbit? Once again, both symbols and J.

(e) How fast (that is, at what ``muzzle velocity'') must you launch
the package (that is sitting on the ground at the equator), and in what direction, if
you want it to leave the gravitational field of the Earth entirely?
Here just give your answer in $\km\,\s^{-1}$.

\paragraph{\problemname~\theproblem:}\refstepcounter{problem}%
\textsl{(a)}~Find some combination of Newton's constant $G$, the speed of light $c$, and
some (arbitrary) mass $M$, that has units of \emph{length}.

\textsl{(b)}~Evaluate your expression for the mass of the Sun (you
will have to look it up); that is, find the characteristic length
associated with that mass.  Give your answer in units of km.  What is
the physical meaning of this length (approximately)?

\paragraph{Extra Problem (will not be graded for credit):}%
Consider the moment when the Sun, Moon, and Earth are (close to)
aligned, with the Moon between the Sun and the Earth. Compute the
force on the Moon from the Earth, and then compute the force on the
Moon from the Sun. Why is the Moon not falling into the Sun? Why was
it that you were permitted to ignore the force from the
Sun in \problemname~\ref{package}, above? It isn't because the force
from the Sun is small!

\paragraph{Extra Problem (will not be graded for credit):}%
Given what you thought about in \problemname~\ref{rocket} above, why do launches
of space vehicles start with the rocket pointing straight up? Why not
point at an angle? There are two important reasons!

\paragraph{Extra Problem (will not be graded for credit):}%
Would it take more or less energy to get a package to escape velocity
at the North Pole relative to what you calculated at the Equator? Make
your argument for a spherical Earth. Now notice that the Earth isn't
exactly spherical. Will that fact increase or decrease the difference?

\end{document}
