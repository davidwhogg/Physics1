\documentclass[12pt]{article}
%% BUGS:
%% - The \begin{problem} ... \end{problem} is only used in exams??

\usepackage{url, graphicx, epstopdf}

% page layout
\setlength{\topmargin}{-0.25in}
\setlength{\textheight}{9.5in}
\setlength{\headheight}{0in}
\setlength{\headsep}{0in}
\setlength{\parindent}{1.1\baselineskip}

% headers
\newcommand{\examheader}[1]{
\noindent
Name:\rule[-1ex]{0.60\textwidth}{0.1pt}
NetID:\rule[-1ex]{0.20\textwidth}{0.1pt}

\section*{{NYU Physics I} --- {#1}}
\setcounter{problem}{1}}

% problem formatting
\newcommand{\problemname}{Problem}
\newcounter{problem}
\newenvironment{problem}{%
  \addvspace{\baselineskip}\noindent\textbf{Problem~\theproblem:}\refstepcounter{problem}
}{%
  \par\addvspace{\baselineskip}
}

% words
\newcommand{\foreign}[1]{\textsl{#1}}
\newcommand{\vs}{\foreign{vs}}

% math
\newcommand{\dd}{\mathrm{d}}
\newcommand{\e}{\mathrm{e}}

% primary units
\newcommand{\rad}{\mathrm{rad}}
\newcommand{\kg}{\mathrm{kg}}
\newcommand{\m}{\mathrm{m}}
\newcommand{\s}{\mathrm{s}}

% secondary units
\renewcommand{\deg}{\mathrm{deg}}
\newcommand{\km}{\mathrm{km}}
\newcommand{\cm}{\mathrm{cm}}
\newcommand{\mm}{\mathrm{mm}}
\newcommand{\ft}{\mathrm{ft}}
\newcommand{\mi}{\mathrm{mi}}
\newcommand{\AU}{\mathrm{AU}}
\newcommand{\ns}{\mathrm{ns}}
\newcommand{\h}{\mathrm{h}}
\newcommand{\yr}{\mathrm{yr}}
\newcommand{\N}{\mathrm{N}}
\newcommand{\J}{\mathrm{J}}
\newcommand{\eV}{\mathrm{eV}}
\newcommand{\W}{\mathrm{W}}
\newcommand{\Pa}{\mathrm{Pa}}

% derived units
\newcommand{\mps}{\m\,\s^{-1}}
\newcommand{\mph}{\mi\,\h^{-1}}
\newcommand{\mpss}{\m\,\s^{-2}}

% random stuff
\sloppy\sloppypar\raggedbottom\frenchspacing\thispagestyle{empty}

\begin{document}

\section*{NYU Physics I---Problem Set 8}

Due Tuesday 2025 October 28 by 23:59 NYC time on Brightspace.
Be sure to give credit to all your sources of help, including your colleagues.
If you use an LLM (or ``GenAI''),
make sure you are consistent with the policy in the syllabus.

\paragraph{Problem~\theproblem:}\refstepcounter{problem}%
You walk at a stride rate that is set, in part, by the natural period of
oscillation of your leg, treated as a pendulum.  Estimate this
period by treating your leg as a massless rod with its entire mass in a point mass at
the end. That is an absurd approximation! But it is okay at the order-of-magnitude
level. Or is it: Is your answer reasonable?

\paragraph{Problem~\theproblem:}\refstepcounter{problem}%
Show that these two descriptions of a simple harmonic oscillator
\begin{equation}
x(t) = A\,\cos(\omega\,t) + B\,\sin(\omega\,t)
\end{equation}
\begin{equation}
x(t) = X\,\cos (\omega\,t+\phi)
\end{equation}
are completely equivalent by finding the relationships between $A, B$
and $X, \phi$ that make them identical.

\paragraph{Problem~\theproblem:}\refstepcounter{problem}%
A very thin ladder of length $L$ and mass $M$ leans against a vertical
wall, on a horizontal floor, making an angle of $\theta$ with respect
to the wall.  Imagine that there is a large coefficient of friction
$\mu$ at the floor so that the ladder is in static
equilibrium, but assume that the wall is effectively frictionless.

\textsl{(a)} Draw a free-body diagram for the ladder, showing all
forces acting.

\textsl{(b)} Using the bottom of the ladder as the axis of rotation or
origin, compute all the forces and torques on the ladder such that it
is in equilibrium.

\textsl{(c)} Why did I make the wall ``effectively frictionless''?

\textsl{(d)} Re-solve the problem using the \emph{top} of the ladder
as the axis of rotation or origin.  What is different in the end?

\textsl{(e)} At what angles $\theta$ would the ladder start to slip?
If $\mu=0.8$ (not unreasonable for a ladder with hard rubber feet on a wood
floor), what is the maximum angle at which you could lean the ladder?

\paragraph{Problem~\theproblem:}\refstepcounter{problem}%
A long, thin rod of length $L$ and cross-sectional area $A$ and
elastic (Young's) modulus $E$ and mass $M$.

\textsl{(a)} Think of the rod as being like a Hooke's Law spring; it
can be stretched by applying a force.  What is the spring constant $k$
for this spring?

\textsl{(b)} By dimensional analysis, can you combine $L$, $A$, $E$,
and $M$ into a frequency $\omega$?  Do you have more than one choice?  If so,
which of the choices makes most sense? That is, think about how your
answer should scale with changes to the problem.

\paragraph{\problemname~\theproblem:}\refstepcounter{problem}%
In Lecture, you saw something like the damped harmonic oscillator
differential equation
\begin{equation}
m\,\frac{\dd^2 x}{\dd t^2} + c\,\frac{\dd x}{\dd t} + k\,x = 0 \quad ,
\end{equation}
where $m$ is the mass, $c$ is a damping coefficient, and $k$ is a
restoring constant (a spring constant).  Here we are going to show
that
\begin{equation}
x(t) = A\,\e^{-\frac{\gamma}{2}\,t}\,\cos (\omega\,t + \phi)
\end{equation}
can be a solution to the differential equation.

\textsl{(a)} What are the SI units for $x$, $t$, $m$, $c$, $k$, $A$, $\gamma$, $\omega$, and $\phi$?

\textsl{(b)} Take a derivative of $x(t)$ to get $v(t)$. Take another
to get $a(t)$.

\textsl{(c)} Now plug your derivatives into the differential equation,
and see if there is a setting of the parameters $\gamma$ and $\omega$
such that the differential equation can be satisfied? \emph{Hint:}
Group sine and cosine terms separately; both sets of terms must sum to
zero for the differential equation to be satisfied. This is related to
the concept of \emph{detailed balance}.

\textsl{(d)} Did you have to assume things about $m, c, k$ to make
your answer work? What things?

\paragraph{Extra Problem (will not be graded for credit):}%
Re-do the previous problem using complex exponentials. That is, assume
\begin{equation}
x(t) = Z\,\e^{\alpha\,t}
\end{equation}
where $Z$ and $\alpha$ are complex numbers. What's different, and
what's the same?

\end{document}
