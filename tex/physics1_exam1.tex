\documentclass[12pt]{article}
\usepackage{url, graphicx, epstopdf}

% page layout
\setlength{\topmargin}{-0.25in}
\setlength{\textheight}{9.5in}
\setlength{\headheight}{0in}
\setlength{\headsep}{0in}
\setlength{\parindent}{1.1\baselineskip}

% problem formatting
\newcommand{\problemname}{Problem}
\newcounter{problem}

% words
\newcommand{\foreign}[1]{\textsl{#1}}
\newcommand{\vs}{\foreign{vs}}

% math
\newcommand{\dd}{\mathrm{d}}
\newcommand{\e}{\mathrm{e}}

% primary units
\newcommand{\rad}{\mathrm{rad}}
\newcommand{\kg}{\mathrm{kg}}
\newcommand{\m}{\mathrm{m}}
\newcommand{\s}{\mathrm{s}}

% secondary units
\renewcommand{\deg}{\mathrm{deg}}
\newcommand{\km}{\mathrm{km}}
\newcommand{\cm}{\mathrm{cm}}
\newcommand{\mm}{\mathrm{mm}}
\newcommand{\ft}{\mathrm{ft}}
\newcommand{\mi}{\mathrm{mi}}
\newcommand{\AU}{\mathrm{AU}}
\newcommand{\ns}{\mathrm{ns}}
\newcommand{\h}{\mathrm{h}}
\newcommand{\yr}{\mathrm{yr}}
\newcommand{\N}{\mathrm{N}}
\newcommand{\J}{\mathrm{J}}
\newcommand{\eV}{\mathrm{eV}}
\newcommand{\W}{\mathrm{W}}
\newcommand{\Pa}{\mathrm{Pa}}

% derived units
\newcommand{\mps}{\m\,\s^{-1}}
\newcommand{\mph}{\mi\,\h^{-1}}
\newcommand{\mpss}{\m\,\s^{-2}}

% random stuff
\sloppy\sloppypar\raggedbottom\frenchspacing\thispagestyle{empty}

\begin{document}\sloppy\sloppypar\raggedbottom\frenchspacing\pagestyle{empty}

\noindent
Name: \rule[-1ex]{0.55\textwidth}{0.1pt}
NetID: \rule[-1ex]{0.2\textwidth}{0.1pt}

\section*{NYU Physics I---Term Exam 1}

\paragraph{Problem~\theproblem:}\refstepcounter{problem}%
We spoke of densities in lecture on 2016-09-08. By what factor,
roughly, is rock more dense than air? No need to be precise.

\vfill

\paragraph{Problem~\theproblem:}\refstepcounter{problem}%
In lecture on 2016-09-13, we had a stone traveling on a parabolic
trajectory (a gravitational trajectory for which air resistance can be
ignored). We drew the velocity vector at two points on the
trajectory. We compared these two vectors. For this problem, name one
property of the velocity vector that changes with time. Name one
property of the velocity vector that stays the same over time.

\vfill

\paragraph{Problem~\theproblem:}\refstepcounter{problem}%
In lecture on 2016-09-20, we drew a free-body diagram for a block on
an inclined plane. What if the plane wasn't frictionless? Draw a new
free-body diagram including friction, and the old free-body diagram
for comparison.

\vfill
~

\clearpage
\paragraph{Problem~\theproblem:}\refstepcounter{problem}%
In Problem Set 1, I asked you about a typical American car traveling
at speed $v$.  The formula $\rho\,A\,v^3$ has units of \emph{power} or
$\W$ or $\J\,\s^{-1}$.  What does this formula say is the power in
$\W$ for a typical American car traveling at speed $60\,\mi\,\h^{-1}\approx
30\,\m\,\s^{-1}$? You will need to make assumptions about $\rho$ and $A$; state them
clearly. And no need to do your arithmetic or calculation precisely, just
get one digit of accuracy.

\vfill

\paragraph{Problem~\theproblem:}\refstepcounter{problem}%
In Problem Set 2, you made a plot of the vertical velocity $v_y$ as a
function of time $t$. Now make a very similar plot, but for a stone
thrown upwards at time $t=0$ at an initial upwards velocity
$v_y=+1.0\,\m\,\s^{-1}$. For simplicity, set the acceleration due to
gravity as $g_y=-10.0\,\m\,\s^{-2}$. Make your plot for the time range
$0<t<1\,\s$. Clearly label the axes, including the beginning and
ending velocity, so everything can be checked quantitatively.

\vfill

\paragraph{Problem~\theproblem:}\refstepcounter{problem}%
In recitation last week, you did a numerical integration on
paper! What change could you have made to make that integration more
accurate? There are multiple possible answers here; choose one.

\vfill
~
\end{document}
