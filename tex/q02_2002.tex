\documentclass[12pt]{article}
\usepackage{graphics}
\begin{document}

\section*{NYU Physics 1---In-class Exam 2}

\vfill

\paragraph{Name:} ~

\paragraph{email:} ~

\paragraph{recitation:} ~

\vfill

This exam consists of two problems.  Write only in this booklet.  Be
sure to show your work.

\vfill ~

\clearpage

\section*{Problem 1}

\noindent~\hfill\includegraphics{../mp/pulley_table.eps}\hfill~

In the system shown, the strings are massless and inextensible and the
pulley is massless and frictionless.  There is a frictional force
$F_\mathrm{f}=\mu\,N$ between block $m_1$ and the table, but when
released from rest, block $m_2$ falls.

(a) What is the acceleration $\vec{a}$ (magnitude and direction) of
mass $m_1$?  Give your answer in terms of quantities shown in the
diagram.

\vfill

(b) When mass $m_2$ has fallen by a distance $\Delta h$, by how much
$\Delta U$ has the potential energy of the system changed?  Has it
increased or decreased?  By how much $\Delta K$ has the kinetic energy
of the system changed?  Has it increased or decreased?  Give your
answers in terms of $\Delta h$ and the symbols shown in the diagram.

\vfill

\emph{Note that you do not have to have solved part (a) correctly to
get part (b).}

\clearpage

\section*{Problem 2}

A physicist of mass $48~\mathrm{kg}$ stands (carefully) on a perfectly
frictionless ice rink, next to a block of ice of mass
$16~\mathrm{kg}$.  Both begin at rest in the ``rink frame.''  The
physicist pushes on the block until it is moving away \emph{relative
to the physicist} at $1~\mathrm{m\,s^{-1}}$.

(a) When viewed in the rink frame, the block ends up moving in the
positive $x$ direction and the physicist ends up moving in the
negative $x$ direction, by conservation of momentum.  What are the
final speeds of the block and physicist, in the rink frame?

\vfill

(b) How much work $W$ did the physicist do?

\vfill ~

\clearpage

[This page intentionally left blank for calculations or other work.]

\end{document}
