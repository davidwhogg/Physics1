\documentclass[12pt]{article}
\usepackage{url, graphicx, epstopdf}

% page layout
\setlength{\topmargin}{-0.25in}
\setlength{\textheight}{9.5in}
\setlength{\headheight}{0in}
\setlength{\headsep}{0in}
\setlength{\parindent}{1.1\baselineskip}

% problem formatting
\newcommand{\problemname}{Problem}
\newcounter{problem}

% words
\newcommand{\foreign}[1]{\textsl{#1}}
\newcommand{\vs}{\foreign{vs}}

% math
\newcommand{\dd}{\mathrm{d}}
\newcommand{\e}{\mathrm{e}}

% primary units
\newcommand{\rad}{\mathrm{rad}}
\newcommand{\kg}{\mathrm{kg}}
\newcommand{\m}{\mathrm{m}}
\newcommand{\s}{\mathrm{s}}

% secondary units
\renewcommand{\deg}{\mathrm{deg}}
\newcommand{\km}{\mathrm{km}}
\newcommand{\cm}{\mathrm{cm}}
\newcommand{\mm}{\mathrm{mm}}
\newcommand{\ft}{\mathrm{ft}}
\newcommand{\mi}{\mathrm{mi}}
\newcommand{\AU}{\mathrm{AU}}
\newcommand{\ns}{\mathrm{ns}}
\newcommand{\h}{\mathrm{h}}
\newcommand{\yr}{\mathrm{yr}}
\newcommand{\N}{\mathrm{N}}
\newcommand{\J}{\mathrm{J}}
\newcommand{\eV}{\mathrm{eV}}
\newcommand{\W}{\mathrm{W}}
\newcommand{\Pa}{\mathrm{Pa}}

% derived units
\newcommand{\mps}{\m\,\s^{-1}}
\newcommand{\mph}{\mi\,\h^{-1}}
\newcommand{\mpss}{\m\,\s^{-2}}

% random stuff
\sloppy\sloppypar\raggedbottom\frenchspacing\thispagestyle{empty}

\begin{document}

\section*{NYU Physics I---spacetime diagrams}

\paragraph{\theproblem}\refstepcounter{problem}%
Draw a spacetime diagram for your own rest-frame.  On the spacetime
diagram, show your own world-line.

\paragraph{\theproblem}\refstepcounter{problem}%
Imagine there is a galaxy flying away from you with a velocity $v =
0.5\,c$. When the galaxy is moving away, it sends back to you a light
signal every $T'=3.336\,\ns$ (as recorded in the galaxy's rest frame).
Draw the world-line of this galaxy on your spacetime diagram and mark
the events corresponding to the departures of the signals from the
galaxy. Draw at least five such events.

\textsl{In order to answer this question, assume that the galaxy is flying away from you
  radially, and set the direction from you to the galaxy to be the $x$ direction. Draw
your diagram in the $x$--$t$ plane.}

\paragraph{\theproblem}\refstepcounter{problem}%
Draw all the world-lines for all the the signals.  Mark the events of
the signals reaching you.

\paragraph{\theproblem}\refstepcounter{problem}%
Calculate the time intervals between the arrival events (arrivals of
the signals from the galaxy) according to you (that is, in your
frame).  Give your answer in terms of $T'$, $\beta$, and $\gamma$.
\textsl{Hint:} It should be longer than what is suggested by the
simple time-dilation formula.

\paragraph{\theproblem}\refstepcounter{problem}%
Why do the time intervals in the previous problem \emph{not} agree
with the time-dilation formula?  What, on the spacetime diagram,
\emph{does} agree with the time-dilation formula?

\end{document}
