\documentclass[12pt]{article}
%% BUGS:
%% - The \begin{problem} ... \end{problem} is only used in exams??

\usepackage{url, graphicx, epstopdf}

% page layout
\setlength{\topmargin}{-0.25in}
\setlength{\textheight}{9.5in}
\setlength{\headheight}{0in}
\setlength{\headsep}{0in}
\setlength{\parindent}{1.1\baselineskip}

% headers
\newcommand{\examheader}[1]{
\noindent
Name:\rule[-1ex]{0.60\textwidth}{0.1pt}
NetID:\rule[-1ex]{0.20\textwidth}{0.1pt}

\section*{{NYU Physics I} --- {#1}}
\setcounter{problem}{1}}

% problem formatting
\newcommand{\problemname}{Problem}
\newcounter{problem}
\newenvironment{problem}{%
  \addvspace{\baselineskip}\noindent\textbf{Problem~\theproblem:}\refstepcounter{problem}
}{%
  \par\addvspace{\baselineskip}
}

% words
\newcommand{\foreign}[1]{\textsl{#1}}
\newcommand{\vs}{\foreign{vs}}

% math
\newcommand{\dd}{\mathrm{d}}
\newcommand{\e}{\mathrm{e}}

% primary units
\newcommand{\rad}{\mathrm{rad}}
\newcommand{\kg}{\mathrm{kg}}
\newcommand{\m}{\mathrm{m}}
\newcommand{\s}{\mathrm{s}}

% secondary units
\renewcommand{\deg}{\mathrm{deg}}
\newcommand{\km}{\mathrm{km}}
\newcommand{\cm}{\mathrm{cm}}
\newcommand{\mm}{\mathrm{mm}}
\newcommand{\ft}{\mathrm{ft}}
\newcommand{\mi}{\mathrm{mi}}
\newcommand{\AU}{\mathrm{AU}}
\newcommand{\ns}{\mathrm{ns}}
\newcommand{\h}{\mathrm{h}}
\newcommand{\yr}{\mathrm{yr}}
\newcommand{\N}{\mathrm{N}}
\newcommand{\J}{\mathrm{J}}
\newcommand{\eV}{\mathrm{eV}}
\newcommand{\W}{\mathrm{W}}
\newcommand{\Pa}{\mathrm{Pa}}

% derived units
\newcommand{\mps}{\m\,\s^{-1}}
\newcommand{\mph}{\mi\,\h^{-1}}
\newcommand{\mpss}{\m\,\s^{-2}}

% random stuff
\sloppy\sloppypar\raggedbottom\frenchspacing\thispagestyle{empty}

\begin{document}

\section*{NYU Physics I---oscillations}

HOGG: FIX THIS SO IT DOESN'T REQUIRE THE SOLUTION TO A DIFF EQ WE HAVEN'T SEEN YET?

Consider a mass $M$ on a spring of spring constant $k$, released from
rest at $t=0$ but from a distance $X$ (in the $x$-direction, which is
parallel to the spring) away from the equilibrium position for the
mass.  Assume there are no other forces acting!

With a partner, do the following graphing assignments. They might
require some planning, because we want to align all the time axes.  In
all the parts below, when we say ``label'' we mean ``give the
horizontal and vertical position of''. That is, we want the graphs to
be quantitatively correct.

\paragraph{\theproblem}\refstepcounter{problem}%
Plot the position of the mass as a function of time for a few periods.
Label minima, maxima, and zero crossings in time, and label the amplitude.

\paragraph{\theproblem}\refstepcounter{problem}%
Plot the velocity of the mass as a function of time for the same time
interval, on a new figure, but time-aligned with the previous plot.  Label
minima, maxima, and zero crossings in time, and label the velocity amplitude.

\paragraph{\theproblem}\refstepcounter{problem}%
Plot the acceleration of the mass as a function of time for the same time
interval, on a new figure, but time-aligned with the previous plot.  Label
minima, maxima, and zero crossings in time, and label the acceleration amplitude.

\paragraph{\theproblem}\refstepcounter{problem}%
Plot the kinetic energy of the mass as a function of time for the same time
interval, on a new figure, but time-aligned with the previous plot.  Label
minima, maxima, and zero crossings (if any) and label the kinetic-energy amplitude.

\paragraph{\theproblem}\refstepcounter{problem}%
Work out the potential energy of a spring stretched by $x$ by
integrating force times distance.

\paragraph{\theproblem}\refstepcounter{problem}%
Plot the potential energy of the mass as a function of time for the same time
interval, on a new figure, but time-aligned with the previous plot.  Label
minima, maxima, and zero crossings (if any) and label the potential-energy amplitude.

\paragraph{\theproblem}\refstepcounter{problem}%
Guess what you will get if you plot the sum of the potential and kinetic energy plots!

\paragraph{\theproblem}\refstepcounter{problem}%
Plot the total of potential plus kinetic energy as a function of time for the same time
interval, on a new figure, but time-aligned with the previous plot.  Label
minima, maxima, and zero crossings (if any) and label the total-energy amplitude.

\paragraph{\theproblem}\refstepcounter{problem}%
Guess what your plots would have looked like if there had been a bit of
damping---a bit of air drag or friction.

\end{document}
