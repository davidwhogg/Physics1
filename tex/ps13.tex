\documentclass[12pt]{article}
\begin{document}
\thispagestyle{empty}

\section*{NYU Engineering Physics 1---Problem set 13}

Due Thursday 2004 April 29 by 4:30pm at Irene Port's office.

\paragraph{Problem 1:}

(a) The Sun is 8.5~kpc (and $1~\mathrm{kpc}=3.1\times
10^{19}~\mathrm{m}$) from the center of our Galaxy, the Milky Way.  It
is travelling on a near-circular orbit at a speed of
$220~\mathrm{km\,s^{-1}}$ (yes, km per second!) around the center of
the Galaxy.  What is the approximate mass of the Milky Way?  Give your
answer in ``solar masses'' $M_\odot$, where $1~M_\odot= 2.0\times
10^{30}~\mathrm{kg}$.

(b) You know the mass of the Earth and the period of the moon's orbit;
what is the distance to the Moon?  Give your answer in m and in AU,
where 1~AU is the average distance from the Sun to the Earth.

(c) Look up the masses of the Sun and Jupiter, and the orbital period
of Jupiter.  Now approximate the Solar System as being comprised of
only the Sun and Jupiter (not a terrible approximation!).  Compute the
distance of the center of mass of the Sun-Jupiter system from the
center of the Sun (in m).  Is this inside or outside the radius of the
Sun?  Compute Jupiter's orbital speed, and then compute what the speed
of the Sun's ``reflex motion'' must be, given that both Jupiter and
the Sun are orbiting around their common center of mass.

\paragraph{Problem 2:}

(a) What is the total mechanical energy of a package of mass $m=
1~\mathrm{kg}$ sitting on the surface of the Earth on the equator?
Take as the zero of potential energy a motionless package at infinity,
and don't forget to include the kinetic energy from the fact that the
package is sitting on the rotating Earth.  Give your answer in J.

(b) To get the package from sitting on the Earth into orbit most
easily, should you launch it to the north, south, east or west?
Explain your reasoning.

(c) What is the total mechanical energy of the package orbiting just
above the surface of the Earth? 

(d) What is the total energy of the package in ``geostationary
orbit'', ie, orbiting at the radius at which one orbit takes 24~h?

(e) How fast must you launch the package, and in what direction, if
you want it to leave the gravitational field of the Earth entirely?

\paragraph{Problem 3:}

Consider the following attractive radial force law, which is
\emph{very different} from Newton's law of gravity:
\begin{equation}
F = \frac{k\,M\,m}{r^4} \quad ,
\end{equation}
where $k$ is a constant.

One of Kepler's laws is that for gravity, orbital period $T$ is
related to orbital radius $r$ by $T\propto r^{3/2}$.  Use either a
dimensional argument or a direct calculation to get the equivalent
relation for this \emph{very different} radial force law.  If you need
to, assume that the orbits are circular.

\paragraph{Problem 4---optional (not for credit):}

A satellite in a circular low-Earth orbit (like the Space Shuttle)
experiences a very weak drag force $F_d$ which opposes its motion
(from residual, high-altitude atmosphere).  This drag force applies a
torque to the satellite's orbit and removes angular momentum.  Compute
the one-orbit change in angular momentum $\Delta L$ (use the torque!)
and change in mechanical energy $\Delta E$ (compute the work!) and
change in orbital radius $\Delta r$ (assume that the orbit remains
circular) due to this drag torque.  Assume that the torque is very
small, so you can make simplifying approximations (like that the
radius change is small relative to the radius, etc).  Now for the
conceptual part: The drag force opposes the motion of the satellite.
Does the satellite speed up or slow down?

\end{document}
