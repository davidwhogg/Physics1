\documentclass[12pt]{article}
\newcommand{\m}{\mathrm{m}}
\newcommand{\cm}{\mathrm{cm}}
\newcommand{\s}{\mathrm{s}}
\renewcommand{\min}{\mathrm{min}}
\renewcommand{\l}{\ell}
\newcommand{\C}{\mathrm{deg\,C}}
\newcounter{problem}
\begin{document}\thispagestyle{empty}

\section*{NYU General Physics 1---Problem set 11}

\paragraph{Problem~\theproblem:}\refstepcounter{problem}%
Consider a string stretched in the $x$ direction, waving transversely
with the $y$ displacement being a function of position $x$ and time $t$ according to
$$ y(x,t) = A\,\cos(\frac{2\pi\,x}{\lambda})\,\cos(\frac{2\pi\,t}{T}) $$
where $A$ is an amplitude, $\lambda$ is the wavelength, and $T$ is
the period.  For definiteness, set $A=1\,\cm$, $\lambda=0.75\,\m$, and
$T=0.25\,\s$.

\textsl{(a)} Draw a picture of $y(t)$ over the time period $0<t<1\,\s$
for the position $x=0.0\,\m$

\textsl{(b)} Draw a picture of $y(x)$ over the spatial interval $0<x<3\,\m$
for the time $t=0.00\,\s$

\textsl{(c)} Draw a picture of $y(x)$ over the spatial interval $0<x<3\,\m$
for the time $t=0.05\,\s$.

\textsl{(c)} Draw a picture of $y(x)$ over the spatial interval $0<x<3\,\m$
for the time $t=0.10\,\s$.

\textsl{(d)} Is the wave moving?  If not, what kind of wave is this?

\textsl{(e)} Draw a picture of $y(x)$ for the time $t = 0.25\,T$.

\textsl{(f)} Take a derivative with respect to time of $y(x,t)$ and then draw a
picture of the $y$-direction velocity $v_y(x)$ for the times $t=0$ and $t =
0.25\,T$.

\paragraph{Problem~\theproblem:}\refstepcounter{problem}%
The bulk modulus (like a Young's modulus, but for isotropic fluids;
look it up) of blood is much higher than the bulk modulus for artery
walls.  What are the implications of this?

\textsl{(a)} The heart modulates the pressure in the blood.  When the
pressure rises, the blood and arteries are subject to increased
stress.  Which distorts more?  The fractional volume of the blood, or
the fractional volume of the artery walls?

\textsl{(b)} The speed of sound in blood is very different from the
speed of sound in the artery walls.  Which do you expect to be higher?
Why?

\textsl{(c)} (optional) Look up values and compute the two speeds of
sound.  This part is optional, because these numbers are hard to find!

\paragraph{Problem~\theproblem:}\refstepcounter{problem}%
You hit a bell with a hammer and it rings middle C.  It starts out
loud (because you just hit it) and then slowly it rings down; that is,
it gets less loud with time.

\textsl{(a)} Why do you think the bell stops ringing?  Where does that
elastic energy go?

\textsl{(b)} Look up the definition of $Q$ or \emph{quality factor}
for an oscillator.  If the bell rings for about three seconds before
it has significantly faded away, what (very roughly) is the $Q$ of the
oscillator?  \emph{Hint: Ignore stuff on the web about ``bandwidth''
  and look for stuff about ``energy dissipated''; these concepts are
  related, but the latter is more comprehensible.}

\textsl{(c)} If you are a geek, look up the highest-$Q$ oscillators
known.  What are the $Q$ factors for these oscillators?

\textsl{(d)} If you have access to a piano, hit the middle-C key hard
and hold it down (to sustain the note).  How long does it take
(roughly) for the note to decrease in volume significantly (by, say, a
factor of a few)?  I am not asking ``how long does it last entirely?''
but instead ``how long does it take to change substantially?''.  What,
roughly, is the $Q$ of the piano string?

\end{document}
