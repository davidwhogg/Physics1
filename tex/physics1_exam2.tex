\documentclass[12pt]{article}
\usepackage{url, graphicx, epstopdf}

% page layout
\setlength{\topmargin}{-0.25in}
\setlength{\textheight}{9.5in}
\setlength{\headheight}{0in}
\setlength{\headsep}{0in}
\setlength{\parindent}{1.1\baselineskip}

% problem formatting
\newcommand{\problemname}{Problem}
\newcounter{problem}

% words
\newcommand{\foreign}[1]{\textsl{#1}}
\newcommand{\vs}{\foreign{vs}}

% math
\newcommand{\dd}{\mathrm{d}}
\newcommand{\e}{\mathrm{e}}

% primary units
\newcommand{\rad}{\mathrm{rad}}
\newcommand{\kg}{\mathrm{kg}}
\newcommand{\m}{\mathrm{m}}
\newcommand{\s}{\mathrm{s}}

% secondary units
\renewcommand{\deg}{\mathrm{deg}}
\newcommand{\km}{\mathrm{km}}
\newcommand{\cm}{\mathrm{cm}}
\newcommand{\mm}{\mathrm{mm}}
\newcommand{\ft}{\mathrm{ft}}
\newcommand{\mi}{\mathrm{mi}}
\newcommand{\AU}{\mathrm{AU}}
\newcommand{\ns}{\mathrm{ns}}
\newcommand{\h}{\mathrm{h}}
\newcommand{\yr}{\mathrm{yr}}
\newcommand{\N}{\mathrm{N}}
\newcommand{\J}{\mathrm{J}}
\newcommand{\eV}{\mathrm{eV}}
\newcommand{\W}{\mathrm{W}}
\newcommand{\Pa}{\mathrm{Pa}}

% derived units
\newcommand{\mps}{\m\,\s^{-1}}
\newcommand{\mph}{\mi\,\h^{-1}}
\newcommand{\mpss}{\m\,\s^{-2}}

% random stuff
\sloppy\sloppypar\raggedbottom\frenchspacing\thispagestyle{empty}

\begin{document}

\noindent
Name: \rule[-1ex]{0.55\textwidth}{0.1pt}
NetID: \rule[-1ex]{0.2\textwidth}{0.1pt}

\section*{NYU Physics I---Term Exam 2}

\paragraph{\problemname~\theproblem:}\refstepcounter{problem}%
(From Lecture on 2016-09-27.) Starting at rest, a package of mass $m$
slides down a ramp and off a jump angled at $45\,\deg$ to the
horizontal. If it has dropped a vertical distance of $H$ by the time
it launches, and if it launches at $45\,\deg$, and if there are no
losses to friction or air resistance, what is the \emph{horizontal
  component} of the velocity at launch? Give an answer in terms of
$m$, $g$, and $H$.

\vfill

\paragraph{\problemname~\theproblem:}\refstepcounter{problem}%
(From Lecture on 2016-10-04.) In a one-dimensional problem, a $4\,\kg$
block moves to the right at $1.5\,\mps$, and another $4\,\kg$ block
moves to the left at $2\,\mps$. Define the positive direction to be
``to the right''. What is the total momentum, and what is the total
kinetic energy?

\vfill

\paragraph{\problemname~\theproblem:}\refstepcounter{problem}%
(From Problem Set 3, problem 1.) A runner on a straight road starts
at rest, accelerates at $5\,\mpss$ for $2\,\s$ and then continues at
constant speed for $10\,\s$. What is the total distance that the
runner has run in the full $12\,\s$ interval?

\vfill
~

\clearpage
\paragraph{\problemname~\theproblem:}\refstepcounter{problem}%
(From Problem Set 4, problem 2.) In the diagram shown at right, assume
that strings are inextensible, and the positive direction is
upwards. Assume that the blocks have nearly equal masses. What is the
relationship, if any, between the acceleration of block $m_1$ and the
acceleration of block $m_2$? No need to solve the whole problem!
\marginpar{\includegraphics[width=1in]{../mp/tackle_blocks.pdf}}

\vfill

\paragraph{\problemname~\theproblem:}\refstepcounter{problem}%
(From blocks-and-pulleys worksheet.) How does it help us to assume
that strings and pulleys are massless? Give one good, specific
equation or reason.

\vfill

\paragraph{\problemname~\theproblem:}\refstepcounter{problem}%
(From friction worksheet.) You have a block of mass $m$ on an inclined
plane, inclined at an angle $\theta=20\,\deg$ to the horizontal. The
coefficient of friction is $\mu=0.9$. What is the magnitude of the
frictional force on the block? The acceleration due to gravity is $g$.

\vfill
~
\end{document}
