\documentclass[12pt]{article}
\usepackage{url, graphicx, epstopdf}

% page layout
\setlength{\topmargin}{-0.25in}
\setlength{\textheight}{9.5in}
\setlength{\headheight}{0in}
\setlength{\headsep}{0in}
\setlength{\parindent}{1.1\baselineskip}

% problem formatting
\newcommand{\problemname}{Problem}
\newcounter{problem}

% words
\newcommand{\foreign}[1]{\textsl{#1}}
\newcommand{\vs}{\foreign{vs}}

% math
\newcommand{\dd}{\mathrm{d}}
\newcommand{\e}{\mathrm{e}}

% primary units
\newcommand{\rad}{\mathrm{rad}}
\newcommand{\kg}{\mathrm{kg}}
\newcommand{\m}{\mathrm{m}}
\newcommand{\s}{\mathrm{s}}

% secondary units
\renewcommand{\deg}{\mathrm{deg}}
\newcommand{\km}{\mathrm{km}}
\newcommand{\cm}{\mathrm{cm}}
\newcommand{\mm}{\mathrm{mm}}
\newcommand{\ft}{\mathrm{ft}}
\newcommand{\mi}{\mathrm{mi}}
\newcommand{\AU}{\mathrm{AU}}
\newcommand{\ns}{\mathrm{ns}}
\newcommand{\h}{\mathrm{h}}
\newcommand{\yr}{\mathrm{yr}}
\newcommand{\N}{\mathrm{N}}
\newcommand{\J}{\mathrm{J}}
\newcommand{\eV}{\mathrm{eV}}
\newcommand{\W}{\mathrm{W}}
\newcommand{\Pa}{\mathrm{Pa}}

% derived units
\newcommand{\mps}{\m\,\s^{-1}}
\newcommand{\mph}{\mi\,\h^{-1}}
\newcommand{\mpss}{\m\,\s^{-2}}

% random stuff
\sloppy\sloppypar\raggedbottom\frenchspacing\thispagestyle{empty}

\begin{document}

\noindent
Name: \rule[-1ex]{0.55\textwidth}{0.1pt}
NetID: \rule[-1ex]{0.2\textwidth}{0.1pt}

\section*{NYU Physics I---Term Exam 2}

\paragraph{\problemname~\theproblem:}\refstepcounter{problem}%
(from Lecture on 2018-09-27)
A roller-coaster cart is at the top of a loop-the-loop (and therefore
upside-down). The trajectory of the center of mass of the cart has a
radius of curvature $R=5\,\m$. How fast does the roller-coaster have
to be moving in $\mps$ to stay on it's proper path (that is, on the
tracks)?  Assume the mass is $M=1000\,\kg$ and the acceleration due to
gravity is $g=10\,\mpss$.

\vfill

\paragraph{\problemname~\theproblem:}\refstepcounter{problem}%
(from Lecture on 2018-09-25)
In \emph{16 words or fewer} tell me why the mass flying off the
(not a) aki jump didn't fly
all the way back up to the release height. Put a box around your answer,
so I can count the words!

\vfill

\paragraph{\problemname~\theproblem:}\refstepcounter{problem}%
(from Problem Set 3)
If a runner, starting at rest, accelerates at $5\,\mpss$ for $2\,\s$
and then continues at constant speed for $19\,s$ more, how far will
she have run at the end of that $21\,\s$?

\vfill
~

\clearpage
\paragraph{\problemname~\theproblem:}\refstepcounter{problem}%
(from Problem Set 4)
What is your kinetic energy when you are walking along the street?
State your assumptions, and make sure they are \emph{reasonable.}

\vfill

\paragraph{\problemname~\theproblem:}\refstepcounter{problem}%
(from the blocks-and-pulleys worksheet)
A massless pulley hangs from the ceiling from a string which is at
tension $T_1$. Over this pulley is another string at tension $T_2$, on
the ends of which are massive blocks attached. What is the
relationship between $T_1$ and $T_2$? If you have to assume additional
things to solve this problem, state them.

\vfill

\paragraph{\problemname~\theproblem:}\refstepcounter{problem}%
(from the friction worksheet)
You have a block of mass $m$ on an inclined plane, inclined at an
angle $\theta=15\,\deg$ to the horizontal. The coefficient of friction
is $\mu=0.9$. What is the magnitude of the frictional force on the
block? The acceleration due to gravity is $g$.
You can leave your answer in terms of $\mu$, $m$, $g$, $\theta$, or
whatever you need to deliver a correct answer.
Once again, state any assumptions you need to make.

\vfill
~
\end{document}
