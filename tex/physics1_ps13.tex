\documentclass[12pt]{article}
\usepackage{url, graphicx, epstopdf}

% page layout
\setlength{\topmargin}{-0.25in}
\setlength{\textheight}{9.5in}
\setlength{\headheight}{0in}
\setlength{\headsep}{0in}
\setlength{\parindent}{1.1\baselineskip}

% problem formatting
\newcommand{\problemname}{Problem}
\newcounter{problem}

% words
\newcommand{\foreign}[1]{\textsl{#1}}
\newcommand{\vs}{\foreign{vs}}

% math
\newcommand{\dd}{\mathrm{d}}
\newcommand{\e}{\mathrm{e}}

% primary units
\newcommand{\rad}{\mathrm{rad}}
\newcommand{\kg}{\mathrm{kg}}
\newcommand{\m}{\mathrm{m}}
\newcommand{\s}{\mathrm{s}}

% secondary units
\renewcommand{\deg}{\mathrm{deg}}
\newcommand{\km}{\mathrm{km}}
\newcommand{\cm}{\mathrm{cm}}
\newcommand{\mm}{\mathrm{mm}}
\newcommand{\ft}{\mathrm{ft}}
\newcommand{\mi}{\mathrm{mi}}
\newcommand{\AU}{\mathrm{AU}}
\newcommand{\ns}{\mathrm{ns}}
\newcommand{\h}{\mathrm{h}}
\newcommand{\yr}{\mathrm{yr}}
\newcommand{\N}{\mathrm{N}}
\newcommand{\J}{\mathrm{J}}
\newcommand{\eV}{\mathrm{eV}}
\newcommand{\W}{\mathrm{W}}
\newcommand{\Pa}{\mathrm{Pa}}

% derived units
\newcommand{\mps}{\m\,\s^{-1}}
\newcommand{\mph}{\mi\,\h^{-1}}
\newcommand{\mpss}{\m\,\s^{-2}}

% random stuff
\sloppy\sloppypar\raggedbottom\frenchspacing\thispagestyle{empty}

\begin{document}

\section*{NYU Physics I---Problem Set 13}

Due Tuesday 2025 December 2 by 23:59 NYC time on Brightspace.
Be sure to cite all your sources as per course policies.

\paragraph{\problemname~\theproblem:}\refstepcounter{problem}%
Get a sense of the speed of light by computing two things:

\textsl{(a)} If you laid out a fiber-optic cable (and we do have many cables that are longer than 1000\,km laid out under oceans),
how many times could light go around the equator of the Earth
in a time interval of $1\,\s$?
If you want to be \emph{extra}, account for the speed of light in glass;
or just use the speed of light in vacuum; we'll take either answer.

\textsl{(b)} How long (in ns) does it take light to go $1\,\ft$?

\paragraph{\problemname~\theproblem:}\refstepcounter{problem}%
Relativity matters most in physics in particle physics experiments (for small things),
and in cosmology (for big things).

\textsl{(a)}
Look up the per-proton energy of the particles in the Large Hadron Collider and divide by the
rest energy ($m\,c^2$) for the proton. That ratio (energy over rest energy) is $\gamma$.
What do you get for $\gamma$, and, for that $\gamma$, what is the corresponding speed $\beta=v/c$ for the
protons? Do you see why we use $\gamma$ and not $\beta$ to describe highly relativistic particles?

\textsl{(b)}
The Universe is $1.4\times 10^{10}\,\yr$ old.
Roughly how far away are the most distant things you can possibly see?
(Note: Distances is cosmology are weird, so this is only approximate.)
Give your answer in SI units and also megaparsecs (Mpc; the units astronomers use).

\textsl{(c)}
The nearest known planet around another star is 4.2\,pc away.
How long would it take for us to get there if we can travel at $30\,\km\,s^{-1}$ (the Earth's orbital speed around the Sun)?
How fast would we have to go to get there in 1~year?
For this question, what I want you to answer is:
At what speed $\beta=v/c$ do you have to travel to make the
(squared) \emph{interval} between the departure event and the arrival event $(\Delta s)^2=(1\,\yr)^2$?

\paragraph{\problemname~\theproblem:}\refstepcounter{problem}%
From the notes at \url{http://cosmo.nyu.edu/hogg/sr/},
Problem 3--4. Note that there is a typo in this part \textsl{(d)} of this
problem: It is the Earth that replies, not the station.
Also, there is no need to draw the spacetime diagram \emph{in the frame of the rocket ship},
but the spacetime diagram in the rest frame of Earth is very useful.

\paragraph{\problemname~\theproblem:}\refstepcounter{problem}%
From the notes at \url{http://cosmo.nyu.edu/hogg/sr/},
Problem 2--14.

\paragraph{\problemname~\theproblem:}\refstepcounter{problem}\label{taylor}%
\textsl{(a)} What is $\gamma$ to first order in $\beta^2$ for $\beta
<< 1$? That is, construct a Taylor Series for $\gamma$ in terms of
$\beta^2$ and give the zeroth-order term (1) and then the first-order
term.

\textsl{(b)} What are $\beta$ and $\gamma$ for a person walking
(relative to the sidewalk), a driver on the freeway (relative to the
road), a commercial jet (relative to the air), and an astronaut in the
ISS (relative to the center of mass of the Earth)? Use the first-order
expression from part \textsl{(a)} to compute the $\gamma$ values.

\textsl{(c)} Computing the full time dilation effect in gravity is
complicated! However, the pure kinematic part of the time dilation
only depends on $\gamma$. Two twins part. One gets on the ISS for a
year, and one stays on Earth. When they are reunited in a year, how
much younger is the astronaut than the homebody?

\paragraph{Extra Problem (will not be graded for credit):}%
If the total energy (rest mass plus kinetic) of a point particle is
$\gamma\,m\,c^2$, use the result from \problemname~\ref{taylor} above to get an
approximate expression for the kinetic energy at low speeds.

\end{document}
