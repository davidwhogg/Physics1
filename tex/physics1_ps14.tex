\documentclass[12pt]{article}
%% BUGS:
%% - The \begin{problem} ... \end{problem} is only used in exams??

\usepackage{url, graphicx, epstopdf}

% page layout
\setlength{\topmargin}{-0.25in}
\setlength{\textheight}{9.5in}
\setlength{\headheight}{0in}
\setlength{\headsep}{0in}
\setlength{\parindent}{1.1\baselineskip}

% headers
\newcommand{\examheader}[1]{
\noindent
Name:\rule[-1ex]{0.60\textwidth}{0.1pt}
NetID:\rule[-1ex]{0.20\textwidth}{0.1pt}

\section*{{NYU Physics I} --- {#1}}
\setcounter{problem}{1}}

% problem formatting
\newcommand{\problemname}{Problem}
\newcounter{problem}
\newenvironment{problem}{%
  \addvspace{\baselineskip}\noindent\textbf{Problem~\theproblem:}\refstepcounter{problem}
}{%
  \par\addvspace{\baselineskip}
}

% words
\newcommand{\foreign}[1]{\textsl{#1}}
\newcommand{\vs}{\foreign{vs}}

% math
\newcommand{\dd}{\mathrm{d}}
\newcommand{\e}{\mathrm{e}}

% primary units
\newcommand{\rad}{\mathrm{rad}}
\newcommand{\kg}{\mathrm{kg}}
\newcommand{\m}{\mathrm{m}}
\newcommand{\s}{\mathrm{s}}

% secondary units
\renewcommand{\deg}{\mathrm{deg}}
\newcommand{\km}{\mathrm{km}}
\newcommand{\cm}{\mathrm{cm}}
\newcommand{\mm}{\mathrm{mm}}
\newcommand{\ft}{\mathrm{ft}}
\newcommand{\mi}{\mathrm{mi}}
\newcommand{\AU}{\mathrm{AU}}
\newcommand{\ns}{\mathrm{ns}}
\newcommand{\h}{\mathrm{h}}
\newcommand{\yr}{\mathrm{yr}}
\newcommand{\N}{\mathrm{N}}
\newcommand{\J}{\mathrm{J}}
\newcommand{\eV}{\mathrm{eV}}
\newcommand{\W}{\mathrm{W}}
\newcommand{\Pa}{\mathrm{Pa}}

% derived units
\newcommand{\mps}{\m\,\s^{-1}}
\newcommand{\mph}{\mi\,\h^{-1}}
\newcommand{\mpss}{\m\,\s^{-2}}

% random stuff
\sloppy\sloppypar\raggedbottom\frenchspacing\thispagestyle{empty}

\begin{document}

\section*{NYU Physics I---Problem Set 14}

Due Thursday 2024 December 12 by 23:59 NYC time on Brightspace.
Be sure to give credit to your sources.

\paragraph{\problemname~\theproblem:}\refstepcounter{problem}%
\textsl{(a)} The kinetic energy of a particle is defined to be the difference between the
total energy $E$ given by
\begin{equation}
E^2 = p^2\,c^2 + m^2\,c^4
\end{equation}
(where $p$ is the relativistically correct momentum formula) and the
rest-mass energy $E_0$ given by
\begin{equation}
E_0 = m\,c^2 \quad.
\end{equation}
Given these formulae, make a table showing the kinetic energy of a $2\,\kg$ mass
moving at the following set of speeds:
\begin{trivlist}
\item $1.0\times10^{-2}\,\mps$ (you might have to use an approximation for this one)
\item $1.0\times10^0\,\mps$ (here too maybe)
\item $1.0\times10^2\,\mps$
\item $1.0\times10^4\,\mps$
\item $1.0\times10^6\,\mps$
\item $1.0\times10^8\,\mps$
\item $2.0\times10^8\,\mps$
\item $2.9\times10^8\,\mps$
\item $0.99\,c$
\item $0.9999\,c$
\end{trivlist}
Give your answers in scientific notation, to seven decimal places, in Joules. Feel free to use
a computer program or spreadsheet to calculate and format this table.

\textsl{(b)} Add to your table a column giving the non-relativistic
formula for kinetic energy we learned many weeks ago ($(1/2)\,m\,v^2$). Do you see any issues
with your calculations, or are you okay with them? (For context, the
first four involve the differences of extremely large, very similar
numbers!) If it needs fixing, can you figure out how to fix it?

\textsl{(c)} Add to your table a column giving the ratio of the
relativistic kinetic energy to the output of the non-relativistic
formula. Again, to 7 decimal places. Feel free to tweak everything until it all looks right.

\paragraph{\problemname~\theproblem:}\refstepcounter{problem}%
From the notes at \url{http://cosmo.nyu.edu/hogg/sr/},
Problem 4--8.

\paragraph{\problemname~\theproblem:}\refstepcounter{problem}%
From the notes at \url{http://cosmo.nyu.edu/hogg/sr/},
Problem 4--11.

\paragraph{\problemname~\theproblem:}\refstepcounter{problem}%
From the notes at \url{http://cosmo.nyu.edu/hogg/sr/},
Problem 6--10.

\paragraph{Extra Problem (will not be graded for credit):}%
\textsl{(a)}~Forgetting about Special Relativity, and assuming just
Newtonian mechanics, compute how long you would have to accelerate at
acceleration $g=10\,\mpss$ in order to reach the speed of light.

\textsl{(b)}~A relativistically correct contstant-acceleration
trajectory on a spacetime diagram is a hyperbola, where both
asymptotes are 45-degree lines (null trajectories. Find a formula
(position $x$ as a function of time $t$) for this hyperbola,
constrained to have acceleration $g$ at small times.

\textsl{(c)}~Show that this trajectory is unchanged under the Lorentz
transformation. That is, show that if you boost in the $x$ direction,
the trajectory translates onto itself (except, possibly, for a small
shift in the $x$ or $t$ direction).

\end{document}
