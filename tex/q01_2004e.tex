\documentclass[12pt]{article}
\usepackage{graphics}
\begin{document}

\section*{NYU Engineering Physics 1---In-class Exam 1}

\vfill

\paragraph{Name:} ~

\paragraph{email:} ~

\vfill

This exam consists of three problems.  Write only in this booklet.  Be
sure to show your work.

\vfill ~

\clearpage

\section*{Problem 1 (3 points)}

In a ten-story building with ten apartments per floor, how much water
is used per day?  Assume that water use is dominated by bathing and
that each apartment fills about one bathtub once per day.  Give your
answer in kg and in $\mathrm{m^3}$.

\clearpage

\section*{Problem 2 (4 points)}

Draw free-body diagrams for all the masses and pulleys in this
mechanism.  Show which forces have equal magnitudes because of
Newton's third law (the action force is equal and opposite to the
reaction force).  Assume that all the pulleys and strings are
massless; assume that each string has a unique tension throughout its
length.\\ \rule{0.35\textwidth}{0pt}
\resizebox{0.3\textwidth}{!}{\includegraphics{../mp/tackle_blocks.eps}}

\clearpage

\section*{Problem 3 (3 points)}

A stone is thrown straight upwards (the positive $y$ direction) from
ground level.  It rises, falls, and hits the ground.  Sketch a graph
of the vertical velocity (the $y$ component of the velocity) of the
stone as a function of time from the throw until hitting the ground.
Label your graph with the point at which the stone is thrown, the
point at which the stone is at its maximum height, and the point at
which the stone hits the ground.  Ignore air resistance.

\end{document}
