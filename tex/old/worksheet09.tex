\documentclass[12pt,letterpaper]{article}
\pagestyle{empty}
\setlength{\headheight}{0in}
\setlength{\headsep}{0in}
\setlength{\topmargin}{-0.5in}
\setlength{\textheight}{10.0in}
\setlength{\oddsidemargin}{-0.5in}
\setlength{\textwidth}{7.5in}
\begin{document}

A mass $m$ on a spring of spring constant $k$ oscillates with
amplitude $A$ (a length).  Give all answers in terms of $m$, $k$, and
$A$.

(a) At what point(s) in the trajectory of the mass is the potential
energy $U(t)$ a maximum, and why, and what is that maximum potential
energy $U_{\mathrm max}$?  A graph can be useful.

\vfill

(b) At what point(s) in the trajectory of the mass is the kinetic
energy $K(t)$ a maximum, and why, and what is that maximum kinetic
energy $K_{\mathrm max}$?

\vfill

(c) Write a general expression for the potential energy $U(t)$ at any
time $t$.  Write another for $K(t)$.  Write one for the total energy
$E(t)$.  You might need an old, familiar trig identity.  Discuss your
answer.

\vfill

(d) Say you double the amplitude $A$.  By what factor will the total
energy $E$ change?

\vfill

(e) Say you reduce the energy in the system by 2 percent.  By what
fraction does the amplitude $A$ reduce?

\vfill

(f) Give an argument---using no calculus---for the limit
$\lim_{\epsilon\rightarrow 0} (1 + \epsilon)^2 = 1 + 2\,\epsilon$.

\vfill

(g) Do you see any relevance of part (f) to part (e)?

\vfill

Discuss

\end{document}
