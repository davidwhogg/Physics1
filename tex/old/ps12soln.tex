\documentclass[12pt]{article}
\usepackage{amssymb}
\usepackage{amsfonts}
\usepackage{epsfig,latexsym}
\voffset -.5cm
\hoffset -1.5cm
\textheight 21cm
\textwidth 16cm
\def\dspace{\baselineskip = .30in}

\def\beq{\begin{equation}}
\def\eeq{\end{equation}}
\def\be{\begin{eqnarray}}
\def\ee{\end{eqnarray}}

\begin{document}
\begin{center}
{\bf\large Homework 12.}
\end{center}

{\bf Problem 1.}

$\Delta E=2\pi E/Q$, where $E=m\omega_0^2 A^2/2$ is the mechanical energy, $A$ is the initial amplitude. If we'd like to have constant amplitude we need pump in each cycle $\Delta E$.
The corresponding power is $P=\Delta E/T=E\omega_0/Q=\gamma E$ (note, since we have 
a motion under periodic force, the  frequency of oscillation is frequency of the force).

For $m=2\;kg$, $T=1\;s$, $A=4\;cm$ and $Q=20$, we get $P=m(2\pi/T)^3 A^2/(2Q)\approx0.02\;Watt$.
\\

{\bf Problem 2.}   See graph.

\begin{figure}[p]
\begin{center}
\epsffile{../eps/h12p2.eps}
\caption{Problem 2.}
\end{center}
\end{figure}


{\bf Problem 3.}

(a) The speed of the wave in the rope is $c=\sqrt{T/\mu}=10\;m/s$.

(b) See graph.
\begin{figure}[h]
\begin{center}
\epsffile{../eps/h122.eps}
\caption{Problem 3 (b,c).}
\end{center}
\end{figure}

(d) All the energy is in the  kinetic form.
\begin{figure}[h]
\begin{center}
\epsffile{../eps/h12d.eps}
\caption{Problem 3 (d).}
\end{center}
\end{figure}

{\bf Problem 4.}

$\tau=I\alpha=-\kappa\theta$, where $\alpha =\ddot{\theta}$ is angular acceleration.
Thus we can rewrite our equation in the form:
$$\ddot{\theta}+\frac{\kappa}{I}\theta=0$$
the solution is $\theta=\theta_0 \cos\left(\omega t +\phi \right)$, where $\omega=\sqrt{\kappa/I}$.
The period is $T=2\pi\sqrt{I/\kappa}$.

If one would change radius of the wheel (rise temperature) it would effect moment of inertia $I\sim R^2$. $T\sim\sqrt{I}\sim R$, thus the period would increase by $0.01\%$. The watch will lose
$10^{-4}\times 3600\;s\times 24\;h\approx 8.6\;s$ a day.




\end{document}
