\documentclass[12pt]{article}
\usepackage{amssymb}
\usepackage{amsfonts}
\usepackage{epsfig,latexsym}
\voffset -2cm
\hoffset -1.5cm
\textheight 20cm
\textwidth 16cm
\def\dspace{\baselineskip = .30in}

\def\beq{\begin{equation}}
\def\eeq{\end{equation}}
\def\be{\begin{eqnarray}}
\def\ee{\end{eqnarray}}


\begin{document}
{\bf Problem 1.}

Centripetal acceleration is given by $a=v^2/R=\omega^2 R$, where $v$ is velocity, $R$ is radius of rotation, $\omega=v/R$ is angular velocity defined as change of angle in unit time
(very useful for rotation problems). The Earth makes one rotation (angle $2\pi$) in 24 hours,
that means $\omega=2\pi/24\;\;\mbox{hours}^{-1}=2\pi/24/3600\;\;s^{-1}$.  Radius of the Earth is
$R=6370\;km\approx 6.4\times10^5\;m$. So, the centripetal acceleration at equator is
$a=\omega^2 R\approx 0.034\;m/s^2 =3.4\times 10^{-3} g$, where $g=9.8\;m/s^2$.
\\

{\bf Problem 2.}

see graphs.
\begin{figure}[p]
\begin{center}
\epsffile{../eps/hogg2gr_gr1.eps}
\epsffile{../eps/hogg2gr_gr2.eps}
\epsffile{../eps/hogg2gr_gr3.eps}
\caption{Problem 2.}
\end{center}
\end{figure}
\\

{\bf Problem 3.}

(i) Constant acceleration $x_0=0$ and $v_0=0$, so $x=a t^2/2$.
(ii) Constant speed $v_a=at_a$\\
Total distance $x_f=at_a^2/2+v_a(t_f-t_a)=at_a^2/2+at_a(t_f-t_a)=at_a t_f -at_a^2/2$.
There are 2 cases: {\bf (a)} $t_a<t_{1/2}$, {\bf (b)} $t_a>t_{1/2}$.

{\bf (a)} $t_a<t_{1/2}$ -- half of the distance is $x_f/2=at_{a}^2/2 + v_a(t_{1/2}-t_{a})$ and
 for total $x_f=at_{a}^2/2 + v_a(t_{f}-t_{a})$ . Substituting $v_a=at_a$ we get
 \be
x_f&=&at_a t_f -at_{a}^2/2 =at_a\left( t_f -t_a/2\right)\\
x_f/2&=&at_a t_{1/2} -at_{a}^2/2 =at_a\left( t_{1/2} -t_a/2\right),
\ee
now let's divide one equation by the other
$$2=\frac{at_a\left( t_f -t_a/2\right)}{at_a\left( t_{1/2} -t_a/2\right)}=\frac{ t_f -t_a/2}{ t_{1/2} -t_a/2}.
$$
From here we get for time of accelerated motion $t_a=2(2t_{1/2}-t_f)=1.84\;s$ and correspondingly for acceleration $a=5.9 \;m/s^2$.

{\bf (b)} $t_a>t_{1/2}$ -- half of the distance is $x_f/2=at_{1/2}^2/2$  and
 for total $x_f=at_{a}^2/2 + v_a(t_{f}-t_{a})$ .  From the first equation we get acceleration.
 From the second we find time $t_a$ (by solving quadratic equation),  but both solution
 are out of range: the first one is less then $t_{1/2}$ , the second is bigger then $t_f$. One
 could see this by making graph $v$ vs. $t$.

 



\end{document}