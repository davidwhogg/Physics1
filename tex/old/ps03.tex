\documentclass[12pt]{article}
\newcommand{\m}{\mathrm{m}}
\newcommand{\s}{\mathrm{s}}
\renewcommand{\d}{\mathrm{d}}
\newcommand{\mo}{\mathrm{mo}}
\newcounter{problem}
\begin{document}
\thispagestyle{empty}

\section*{NYU Physics 1---Problem set 3}

Due Tuesday 2009 October 6 at the beginning of lecture.

\paragraph{Problem~\theproblem:}\refstepcounter{problem}%
Chabay \& Sherwood problem 3.P.33.

\paragraph{Problem~\theproblem:}\refstepcounter{problem}%
Imagine a particle of mass $m$ moving back and forth in the $x$
direction according to the equation
\begin{equation}
x(t)= A\,\cos\left(\omega\,t\right) \quad ,
\end{equation}
where $A$ and $\omega$ are constants.  \textsl{(a)}~What are the SI
units for $A$ and $\omega$?  \textsl{(b)}~Derive an exact expression
for the $x$ component $v_x(t)$ of the velocity as a function of time.
\textsl{(c)}~Derive an exact espression for the $x$ component $F_x(t)$
of the net force acting on the particle as a function of time.
\textsl{(d)}~What is the ratio $F_x/x$ as a function of time?

\paragraph{Problem~\theproblem:}\refstepcounter{problem}%
The example problem on pages 85 and 86 of Chabay \& Sherwood gives
masses for the Earth and Sun and an initial position and velocity for
the Earth in its orbit.  Create a spreadsheet or computer program that
integrates the same trajectory as that integrated in the example
problem, but use $1\,\d$ ($8.64\times 10^4\,\s$) time steps instead of
$1\,\mo$ time steps.

I recommend that your spreadsheet or program keep track of the time
$t$, the $x$ component of position $x$, the $y$ component of position
$y$, the magnitude of the position $r=\sqrt{x^2+y^2}$, the $x$
component of momentum $p_x$, the $y$ component of momentum $p_y$, the
magnitude of the force $F$, the $x$ component of the force $F_x$, and
the $y$ component of the force $F_y$.  You can test that you have your
formulae correct by changing the timestep from $1\,\d$ up to $1\,\mo$
and seeing that you reproduce all of the numbers in the textbook
example problem (though check the errata referred to on the web site
for the course).

Integrate your orbit for $100\,d$ and hand in a plot of $x$ vs $y$ for
those 100 days.  Define the time on the first line of your spreadsheet
$t=0$.  On what day does the orbit cross the $y$ axis?  That is, on
what day does the $x$ position become negative?

\paragraph{Extra Problem (will not be graded for credit):}
Imagine you plan a interstellar voyage in which your space ship
accelerates at $g=10~\m\,\s^{-2}$ for two years.  What is wrong with
your plan?  Now imagine you increase your momentum at a constant rate
of momentum increase (that is, constant force) for two years, with
your force set so that in the first few days (when you are not moving
fast), you are accelerating at $g$.  What speed do you end up going at
the end of the two-year period, relative to the frame at which you
were at rest when you started?  \textit{Note: This problem is very
  similar to the the problem done in lecture.}

Recall that the change in momentum is the force times the time, and
recall that momentum is $\gamma\,m\,v$.  You will have to assume that
your spacecraft has some mass $m$, but you will find that the mass
cancels out in your final answer.

\end{document}
