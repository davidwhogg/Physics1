\documentclass[12pt,letterpaper]{article}
\pagestyle{empty}
\setlength{\headheight}{0in}
\setlength{\headsep}{0in}
\setlength{\topmargin}{-0.5in}
\setlength{\textheight}{10.0in}
\setlength{\oddsidemargin}{-0.5in}
\setlength{\textwidth}{7.5in}
\begin{document}

A typical American car gets about $25~\mathrm{mi\,gal^{-1}}$, driving
at a constant speed of $55~\mathrm{mi\,h^{-1}}$ on level ground.
Since the car is neither climbing nor accelerating, the power put out
by the motor is simply fighting friction and air resistance (drag).

(a) How much energy in J is there in a gallon of gas, approximately?

Useful facts include: (1)~the molecular weight of gasoline is roughly
$100~\mathrm{amu}$ (ie, a typical gasoline molecule is about 100 times
the mass of a hydrogen atom); (2)~gasoline is slightly less dense than
water (but not far less dense); (3)~the typical amount of energy
released when a single gasoline molecule is burned is a few eV per
carbon atom, or, say $5\times 10^{-19}~\mathrm{J}$ per carbon atom.
What fraction of the mass of gasoline is the carbon, do you think?

\vfill

(b) If a typical car is only about 25~percent efficient in its
conversion of chemical energy into mechanical energy, compute the
mechanical power (energy per time) being exerted against friction and
drag at $55~\mathrm{mi\,h^{-1}}$ in horsepower (about 750~W is 1~hp).

\vfill

(c) What is the relationship between power, force, and speed?  Use
dimensional analysis.

\vfill

(d) Justify your answer to part (c) with a physical argument.

\vfill

(e) Compute the force $F$ that the car ``must'' be providing to be
using up all that gasoline.  Give your answer in Newtons.

\vfill

(f) Now compute the force $F$ you expect just given that air
resistance is the main source of the force, and air resistance depends
only on the mass density of air $\rho$, the speed $v$, and the
cross-sectional area of the car $A$.  Figure out what the force $F$
must be on dimensional grounds in terms of these quantities.

\vfill

(g) Plug in realistic numbers for $\rho$, $A$, and $v$ and compare to
your answer in part (e).

\vfill

Explain any discrepancy!

\end{document}
