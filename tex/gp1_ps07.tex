\documentclass[12pt]{article}
\usepackage[pdftex]{graphicx}
\newcommand{\kg}{\mathrm{kg}}
\newcommand{\m}{\mathrm{m}}
\newcommand{\mi}{\mathrm{mi}}
\newcommand{\s}{\mathrm{s}}
\newcommand{\h}{\mathrm{h}}
\newcommand{\mps}{\m\,\s^{-1}}
\newcommand{\J}{\mathrm{J}}
\newcommand{\eV}{\mathrm{eV}}
\newcounter{problem}
\begin{document}\thispagestyle{empty}

\section*{NYU General Physics 1---Problem set 7}

\paragraph{Problem~\theproblem:}\refstepcounter{problem}%
Gasoline and olive oil are both substances with great chemical energy
content per unit mass.

\textsl{(a)} In the case of gasoline, the chemical energy is mainly in
carbon bonds.  If you assume that gasoline is \emph{entirely} carbon
atoms, and each one releases $4\,\eV$ of energy when it is combusted, how
much energy per unit mass is there in gasoline?  Get an answer in MJ
per kg and compare to what you find on \textit{Wikipedia}.  How far off
are our assumptions?

\textsl{(b)} Now convert your answer to kcal per g and compare it to
what is written on the ``Nutrition Facts'' label on an olive oil
bottle.  How close are you?  It should be close, I think, because
biofuel is made from things like olive oil!

\textsl{(c)} Now assume that a car moving at speed
$v=75\,\mi\,\h^{-1}$ encounters an air resistance force of
$\rho\,A\,v^2$, where $\rho$ is the density of air and $A$ is the
cross-sectional area of the car, about $2\,\m^2$ How much work does it
take to move the car $30\,\mi$ at this speed?

\textsl{(d)} If a car with these properties was \emph{perfectly
  efficient}, how many miles per gallon would it get?  What does this
make you think about the future of energy-efficient cars?

\paragraph{Problem~\theproblem:}\refstepcounter{problem}%
A (magical) perfect ball of mass $m=0.7\,\kg$ is dropped from a height
$h=0.9\,\m$ onto a hard surface, off of which it bounces perfectly.
It then continues to bounce forever.  Make a plot, labeling carefully
the time and energy axes, of the kinetic energy of the ball as a function of
time, and of the gravitational potential energy of the ball as a
function of time.  Assume that the bounces (the times in contact with
the floor) are extremely short (negligible).  Now also plot the sum of
the kinetic energy and the potential energy (the total energy) as a
function of time.  For all three plots, label relevant times and
energies.  You have to make a choice about the ``zero'' of potential
energy, right?

For extra fun: Make the bounces last a short time $\Delta t$ and plot
also the ``elastic'' potential energy in the ball!

Sanity check: This problem is impossible because in real life, each
bounce will be shorter than the previous bounce.  Why?  Where does
that energy go?

\paragraph{Problem~\theproblem:}\refstepcounter{problem}%
A New York City Bus moving down Broadway at $15\,\mps$ hits a small
elastic rubber ball, which has happens to be very close to at rest
just before the collision.  The bus has a mass of $2\times 10^4\,\kg$
and the ball has a mass of $0.02\,\kg$.

\textsl{(a)} Draw a diagram showing the bus and the ball and their
velocities, immediately \emph{prior} to the collision.  Compute the
kinetic energies and the momenta of the bus and the ball.  For the
momenta, you will have to choose a direction for your coordinate
system.

\textsl{(b)} Draw the same diagram, but now from the point of view of
the driver of the truck next to the bus, who is driving at the exact
same speed.  That is, draw the diagram in the ``reference frame'' in
which the bus is at rest before the collision.  In this frame, the bus
is stationary and the ball is moving.  Again, compute the kinetic
energies and momenta.  Why are they different from what you got in
part \textsl{(a)}?  Aren't energy and momentum conserved?

\textsl{(c)} Staying in this new reference frame, imagine now
how the ball bounces off the bus.  The collision will be elastic, but
you don't really have to calculate anything: What happens when a tiny
ball bounces elastically off a huge bus?  Draw a diagram showing the
bus and the ball and their velocities, in the frame of the bus,
immediately \emph{after} the collision.  Compute the kinetic energies
and momenta.

\textsl{(d)} If you made a certain very useful approximation, then you
probably didn't conserve momentum in part \textsl{(c)}.  Why not?  In
detail, the velocity of the bus is affected by the collision.  By how
much does the velocity of the bus change, approximately, in the
collision?  If you \emph{did} conserve momentum in part \textsl{(c)},
then just report here the change in velocity of the bus.

\textsl{(e)} Now take what you had in part \textsl{(c)} and re-draw it
back in the original reference frame, which is that of the stores on
Broadway.  Compute the kinetic energies and momenta.  How fast is the
rubber ball moving after the collision?  Don't try the experiment.

\textsl{(f)} After the collision, the ball is moving fast, but its
mass is much smaller than that of the bus.  At the end of the problem,
what fraction of the total system momentum and kinetic energy are in
the ball?

\end{document}
