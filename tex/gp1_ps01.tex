\documentclass[12pt]{article}
\newcommand{\kg}{\mathrm{kg}}
\newcommand{\m}{\mathrm{m}}
\newcommand{\s}{\mathrm{s}}
\begin{document}
\newcounter{problem}
\thispagestyle{empty}

\section*{NYU General Physics 1---Problem set 1}

\paragraph{Problem~\theproblem:}\refstepcounter{problem}%
\textsl{(a)}~What is the maximum amount of cash you can obtain by
successfully robbing an armored truck?  Assume that it is packed with
twenties; that is, estimate an answer by estimating the likely volume
of the inside of the truck, and (harder) the volume of a 20-dollar
bill.  State your assumptions and explain your work, but please don't
attempt an experiment.  Be sure to explain exactly how you estimated
the volume of a 20-dollar bill.  \emph{Hint: think of a stack of bills
  to estimate the volume.  Feel free to \emph{check} any part of your
  answer on the internet, but make sure you actually make a justified,
  quantitative estimate independently.}

\textsl{(b)}~Do you think that many of the armored trucks in Manhattan
are fully packed with 20-dollar bills?

\textsl{(c)}~Would a similar truck weigh more, less, or about the same
if it contained the same amount of money but in the form of gold bars
instead of 20-dollar bills?

\paragraph{Problem~\theproblem:}\refstepcounter{problem}%
\textsl{(a)}~Power $P$ has dimensions of energy per time.  Energy $E$
has dimensions you can infer from the famous formula $E=m\,c^2$.  What
combination can you make of a speed $v$ (length per time), an area $A$
(length squared), and a density $\rho$ (mass per volume) that has
dimensions of power?

\textsl{(b)}~Can you think of any environmental or economic or
engineering significance of that calculation?  \emph{Hint: Imagine
  that the density is the density of the air, and the area is the
  cross-sectional area of a car!}

\textsl{(c)}~What combination can you make of a speed $v$, a mass $m$,
and an acceleration $g$ (length per time squared) that has dimensions
of power?

\textsl{(d)}~Can you think of any biological or physiological
significance of that calculation?  \emph{Hint: Imagine that the speed
  is the speed you can sustainably climb stairs, the mass is your
  mass, and the acceleration is the acceleration due to gravity!}

\paragraph{Problem~\theproblem:}\refstepcounter{problem}%
For very small objects, like cells in a centrifuge, there is a
``settling velocity'' that depends on a density $\rho$ (actually a
density difference, but we will worry about that much later), an
acceleration $g$, a particle size (length or radius) $R$, and a
``dynamic viscosity'' $\mu$.  The dynamic viscosity has dimensions
of mass per length per time (for example, $\kg\,\m^{-1}\,\s^{-1}$; if that seems odd, look it up).  Without
worrying about any details---this is only dimensional analysis---find
a combination of these quantities that has the dimensions of velocity
(length per time).  Compare to what is written in the first couple of
paragraphs of the Wikipedia page called ``Stokes Law''.

\end{document}

\paragraph{Problem~\theproblem:}\refstepcounter{problem}%
Roughly how far does a typical American car have to drive to ``sweep
up'' or drive through a column of air that is comparable in weight to
the car itself?  You will have to estimate the cross-sectional area
and weight of a typical car (or look both things up on the web; if you
look them up, give the make and model).

