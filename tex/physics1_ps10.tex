\documentclass[12pt]{article}
%% BUGS:
%% - The \begin{problem} ... \end{problem} is only used in exams??

\usepackage{url, graphicx, epstopdf}

% page layout
\setlength{\topmargin}{-0.25in}
\setlength{\textheight}{9.5in}
\setlength{\headheight}{0in}
\setlength{\headsep}{0in}
\setlength{\parindent}{1.1\baselineskip}

% headers
\newcommand{\examheader}[1]{
\noindent
Name:\rule[-1ex]{0.60\textwidth}{0.1pt}
NetID:\rule[-1ex]{0.20\textwidth}{0.1pt}

\section*{{NYU Physics I} --- {#1}}
\setcounter{problem}{1}}

% problem formatting
\newcommand{\problemname}{Problem}
\newcounter{problem}
\newenvironment{problem}{%
  \addvspace{\baselineskip}\noindent\textbf{Problem~\theproblem:}\refstepcounter{problem}
}{%
  \par\addvspace{\baselineskip}
}

% words
\newcommand{\foreign}[1]{\textsl{#1}}
\newcommand{\vs}{\foreign{vs}}

% math
\newcommand{\dd}{\mathrm{d}}
\newcommand{\e}{\mathrm{e}}

% primary units
\newcommand{\rad}{\mathrm{rad}}
\newcommand{\kg}{\mathrm{kg}}
\newcommand{\m}{\mathrm{m}}
\newcommand{\s}{\mathrm{s}}

% secondary units
\renewcommand{\deg}{\mathrm{deg}}
\newcommand{\km}{\mathrm{km}}
\newcommand{\cm}{\mathrm{cm}}
\newcommand{\mm}{\mathrm{mm}}
\newcommand{\ft}{\mathrm{ft}}
\newcommand{\mi}{\mathrm{mi}}
\newcommand{\AU}{\mathrm{AU}}
\newcommand{\ns}{\mathrm{ns}}
\newcommand{\h}{\mathrm{h}}
\newcommand{\yr}{\mathrm{yr}}
\newcommand{\N}{\mathrm{N}}
\newcommand{\J}{\mathrm{J}}
\newcommand{\eV}{\mathrm{eV}}
\newcommand{\W}{\mathrm{W}}
\newcommand{\Pa}{\mathrm{Pa}}

% derived units
\newcommand{\mps}{\m\,\s^{-1}}
\newcommand{\mph}{\mi\,\h^{-1}}
\newcommand{\mpss}{\m\,\s^{-2}}

% random stuff
\sloppy\sloppypar\raggedbottom\frenchspacing\thispagestyle{empty}

\begin{document}

\section*{NYU Physics I---Problem Set 10}

Due Tuesday 2025 November 11 at 23:59 NYC time. Be sure to cite your sources and helpers,
consistent with our credit policy and our LLM policy.

\paragraph{\problemname~\theproblem:}\refstepcounter{problem}\label{cartorque}%
\textsl{(a)}~A car of mass $M$ is moving at speed $v$ in the $x$
direction. Its center of mass is a height $h$ above the ground.  What
is the angular momentum of the car with respect to a reference point
\emph{on the ground}?

\textsl{(b)}~If this same car is accelerating at acceleration $a$
in the $x$ direction, then its angular momentum is changing with time,
right? If so, there must be a net torque on the car? What must be
the magnitude of that net torque? Again, answer this with respect
to a reference point \emph{on the ground}.

\paragraph{\problemname~\theproblem:}\refstepcounter{problem}%
In Lecture we considered an ice cube floating in a glass of water at
0\,C.  The ice cube melts at constant temperature, such that you end
up with a glass of water at 0\,C. As it melts, does the water level go
up, or go down, or stay the same? Give an explanation in words (not
equations) that is \emph{shorter than 100 words}.

\textsl{For this problem, you ought to ignore any pressure gradient in
  the air above the water. There is a tiny correction from this air
  gradient, but it is (a) very tiny, and (b) complicating.}

\paragraph{\problemname~\theproblem:}\refstepcounter{problem}%
\textsl{(a)}~A figure skater spins in place on frictionless ice at
angular speed $\omega_i$ with her hands outstretched.  She has a total
moment of inertia $I_i$.  As the skater draws her hands into her body,
her moment of inertia decreases to $I_f=I_i/2$.  Does her kinetic
energy $K$ increase, decrease, or stay the same?  If it increases,
where does the energy come from?  If it decreases, where does the
energy go to?  \emph{Explain all your answers concisely but clearly:
What is conserved? That is, think in terms of conserved quantities.}

\textsl{(b)}~Now estimate the moments of inertia: $I_i$ of an ice
skater with her hands outstretched, and $I_f$ of an ice skater with
her hands drawn in.  Is the factor of 2 used in part \textsl{(a)}
reasonable?

\paragraph{\problemname~\theproblem:}\refstepcounter{problem}\label{cue}%
\textsl{(a)}~Immediately after being hit, at $t=0$, a cue ball of mass
$M$ and radius $R$ slides along the felt at speed $v_i$, not rotating
at all.  As time goes on, the ball slows down (because of friction)
and, at the same time, starts to spin.  Draw a free-body diagram for
the cue ball.  At what time $t_\mathrm{r}$ does the ball get to the
situation of ``rolling without slipping''?  Assume that there is a
coefficient $\mu$ of sliding friction. You will have to look up (or
compute) the moment of inertia $I$ for a uniform sphere.

\textsl{(b)}~Plot $v(t)$ and $R\,\omega(t)$ vs $t$ on a single plot.
\emph{Note that the two things I have asked you to plot have the same
dimensions.}  Clearly label $t_\mathrm{r}$ on your diagram.

\paragraph{\problemname~\theproblem:}\refstepcounter{problem}%
A sticky ball (assume a uniform solid sphere) of mass $m$ and radius
$R$ is moving initially in the $x$ direction at speed $v_0$. It
collides with an identical sticky ball of mass $m$ and radius $R$ that
is initially at rest.  Neither of the two balls is spinning prior to
the collision.  After the collision, the two balls are stuck together
and moving together as one object of mass $2\,m$ and a weird shape.

\textsl{(a)} What is the moment of inertia of an object made up of two
spherical balls of mass $m$ and radius $R$ stuck together at their
surfaces (the ``stuck-together object'')?

\textsl{(b)} Imagine that the two balls collide exactly head on.  What
is the final speed $v$ and rotation rate $\omega$ of the
stuck-together object?

\textsl{(c)} Now imagine that they collide \emph{not} head-on, but at
a $y$-direction offset of $b$. That is, they have an \emph{impact
parameter} of $b$. What is the final speed $v$ and rotation rate
$\omega$ of the stuck-together object both as a function of $b$?
\textsl{Important: In answering this part, be sure to choose clearly
  your reference point (around which angular momenta are calculated),
  and be very clear about what spin direction corresponds to positive
  angular momentum.}

\textsl{(d)} What is the total kinetic energy ``lost'' in the
collision? That is, if you total up linear and spin kinetic energy
before, and linear and spin kinetic energy after, what is the
fractional difference (before minus after, all divided by after) as a
function of $b$.  Does your answer make sense?

\paragraph{Extra Problem (will not be graded for credit):}%
In \problemname~\ref{cue}, between the initial hit of the cue ball by the cue (that is, when the
cue ball wasn't rotating at all) and the end of the cue ball
slide (that is, when the cue ball switches to rolling without
slipping), how much rotational kinetic energy ($I\,\omega^2 / 2$) was
created? How much linear kinetic energy ($m\,v^2 / 2$) was lost? How
much heat was generated? Do these energies relate in any geometric way
to the graph you ended up drawing?

\paragraph{Extra Problem (will not be graded for credit):}%
Take your answer to \problemname~\ref{cartorque}, plug in reasonable
numbers for a good car accelerating from rest, and compare to
published torques from car manufacturers. Makes sense?


\end{document}
