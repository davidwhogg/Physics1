\documentclass[12pt]{article}
\usepackage{url, graphicx, epstopdf}

% page layout
\setlength{\topmargin}{-0.25in}
\setlength{\textheight}{9.5in}
\setlength{\headheight}{0in}
\setlength{\headsep}{0in}
\setlength{\parindent}{1.1\baselineskip}

% problem formatting
\newcommand{\problemname}{Problem}
\newcounter{problem}

% words
\newcommand{\foreign}[1]{\textsl{#1}}
\newcommand{\vs}{\foreign{vs}}

% math
\newcommand{\dd}{\mathrm{d}}
\newcommand{\e}{\mathrm{e}}

% primary units
\newcommand{\rad}{\mathrm{rad}}
\newcommand{\kg}{\mathrm{kg}}
\newcommand{\m}{\mathrm{m}}
\newcommand{\s}{\mathrm{s}}

% secondary units
\renewcommand{\deg}{\mathrm{deg}}
\newcommand{\km}{\mathrm{km}}
\newcommand{\cm}{\mathrm{cm}}
\newcommand{\mm}{\mathrm{mm}}
\newcommand{\ft}{\mathrm{ft}}
\newcommand{\mi}{\mathrm{mi}}
\newcommand{\AU}{\mathrm{AU}}
\newcommand{\ns}{\mathrm{ns}}
\newcommand{\h}{\mathrm{h}}
\newcommand{\yr}{\mathrm{yr}}
\newcommand{\N}{\mathrm{N}}
\newcommand{\J}{\mathrm{J}}
\newcommand{\eV}{\mathrm{eV}}
\newcommand{\W}{\mathrm{W}}
\newcommand{\Pa}{\mathrm{Pa}}

% derived units
\newcommand{\mps}{\m\,\s^{-1}}
\newcommand{\mph}{\mi\,\h^{-1}}
\newcommand{\mpss}{\m\,\s^{-2}}

% random stuff
\sloppy\sloppypar\raggedbottom\frenchspacing\thispagestyle{empty}

\newcommand{\ap}{\mathrm{ap}}
\newcommand{\peri}{\mathrm{peri}}
\begin{document}

\section*{NYU Physics I---Problem Set 12}

Due Thursday 2018 November 29 at the beginning of lecture.

\paragraph{\problemname~\theproblem:}\refstepcounter{problem}%
What is the most expensive ingredient of a typical, traditional
Thanksgiving dinner \emph{by weight} (that is, in dollars per ounce or
per pound).  Start with the turkey and show your work (that is,
compare some ingredients). Don't forget the trace (that is, small
in quantity) ingredients!
What is the relevance of all this to world
history? Keep it traditional---traditional food with traditional
ingredients, like you could have cooked in 1850.
You might want to discuss with someone who cooked
a Thanksgiving dinner (or did the shopping for it).

\paragraph{\problemname~\theproblem:}\refstepcounter{problem}%
If all goes well in class on 2018-11-20, we will get a quadratic equation
for the radii $r_\ap$ and $r_\peri$ of aphelion and perihelion. The argument
goes like this: The total energy of an orbit can be written in terms of the
angular momentum and the radial velocity
\begin{eqnarray}
E & = & \frac{1}{2}\,m\,v^2 - \frac{G\,M\,m}{r}
\\
E & = & \frac{1}{2}\,m\,v_r^2 + \frac{1}{2}\,m\,v_\perp^2 - \frac{G\,M\,m}{r}
\\
E & = & \frac{1}{2}\,m\,v_r^2 + \frac{L^2}{2\,m\,r^2} - \frac{G\,M\,m}{r}
\label{foo}
\end{eqnarray}
where $E$ is the total energy of the orbit, $v_r$ is the radial component of the velocity,
and $L$ is the angular momentum of the orbit.
The radial velocity $v_r$ is exactly zero at aphelion and perihelion.
So set it to zero in equation~(\ref{foo}),
and solve the resulting quadratic equation! It will give two answers,
which are $r_\ap$ and $r_\peri$.
Use the definition of eccentricity
\begin{equation}
e \equiv \frac{r_\ap - r_\peri}{r_\ap + r_\peri}
\end{equation}
to figure out
the relationship between eccentricity $e$ of an elliptical orbit and
the total energy $E$ and the angular momentum $L$.

\paragraph{\problemname~\theproblem:}\refstepcounter{problem}%
\textsl{(a)}
Sketch orbits of fixed semi-major axis but increasing
eccentricity, from a circular orbit, to one that is close to radial
(eccentricity close to unity). Make sure you show the location of
the point around which the object is orbiting!

\textsl{(b)}
What is the transfer time for a radial plunge orbit
from the radius of the Moon's orbit down to the surface of the Earth?
Use the period of the Moon's orbit, the relevant  Kepler's law, and
the properties of the unit-eccentricity and circular orbits.

\textsl{(c)}
Look up the timeline of the Apollo~11 mission, especially
the return to Earth.  Do you see any issues there? What's your best
explanation of what happened?

\paragraph{\problemname~\theproblem:}\refstepcounter{problem}%
\textsl{(a)}
How fast do you have to move with respect to the Earth's surface to
escape Earth's gravity? That is, what is escape velocity from the Earth.
Calculate it yourself in terms of the radius $R$ of the Earth and the
value of $g$ at the surface. Then give it also in $\mps$.

\textsl{(b)} A spaceship of mass $m$ resting on the surface of the
Earth is bound to the Earth but also to the Sun. If we make
the naive (and close to correct) assumption that these energies just
add, what is the total binding energy of the spaceship in the Solar
System? This calculation can be confusing, because although you can assume
the spaceship is stationary with respect to the Earth (so there is
only gravitational potential energy with respect to the Earth), the
spaceship is moving fast relative to the Sun (so there is both
gravitational potential and kinetic energy with respect to the
Sun). The best way to do the calculation is to just pick the Newtonian reference
frame centered on the Sun, and compute the kinetic and potential
energies in that frame. Now what is the escape velocity from the Solar
System? Give your answer in $\mps$.

\textsl{(c)} Look up the derivation of how a rocket accelerates. You
should be able to find a rocket equation that relates the initial mass
of the rocket+fuel, the final mass of the rocket after the fuel is
spent, the speed at which the rocket ejects exhaust, and the final
speed of the rocket. (Hint: The equation is exponential in a mass
ratio.) If the rocket can eject exhaust at 10 times the speed of sound
in air at STP (and that's optimistic!), what is the ratio of initial
mass to final mass of a rocket that will leave the Solar System? What
is the maximum fraction of the spaceship initial mass that can be used
for payload---that is, for cabin, crew, and cargo?

\textsl{(d)} Now imagine the spaceship is going to another planet just
like the Earth. What fraction of the spaceship can be used for
non-fuel payload in this case? The point is that it takes just as much
velocity change to slow down at the end of the journey as it took to
take off at the beginning, and that the end-of-flight fuel is part of
the cargo that the ship has to take with it at launch. You should get
that the payload fraction is (something like) the square of what you got in
the previous part.

\paragraph{Extra Problem (will not be graded for credit):}%
What do you think the above problem means for interstellar travel?

\paragraph{Extra Problem (will not be graded for credit):}%
Make a spreadsheet integration that integrates a test-particle
(low-mass) orbit in the central force law, and show that you do indeed
get an elliptical orbit. You want to use a time step that is $<0.001$
of the period and go for $>1000$ timesteps if you want the integration
to look good! This integration is harder than other integrations you
have done in this class, because you have to project the force onto
the $x$ and $y$ directions correctly. That is, you have to keep track
of both $x$ and $y$ positions, velocities, and accelerations. Do the
problem in the two-dimensional plane of the orbit. If you want
feedback or get stuck, bring your intermediate work to Prof~Hogg for
discussion.

\end{document}
