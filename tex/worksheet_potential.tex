\documentclass[12pt]{article}
\usepackage{url, graphicx, epstopdf}

% page layout
\setlength{\topmargin}{-0.25in}
\setlength{\textheight}{9.5in}
\setlength{\headheight}{0in}
\setlength{\headsep}{0in}
\setlength{\parindent}{1.1\baselineskip}

% problem formatting
\newcommand{\problemname}{Problem}
\newcounter{problem}

% words
\newcommand{\foreign}[1]{\textsl{#1}}
\newcommand{\vs}{\foreign{vs}}

% math
\newcommand{\dd}{\mathrm{d}}
\newcommand{\e}{\mathrm{e}}

% primary units
\newcommand{\rad}{\mathrm{rad}}
\newcommand{\kg}{\mathrm{kg}}
\newcommand{\m}{\mathrm{m}}
\newcommand{\s}{\mathrm{s}}

% secondary units
\renewcommand{\deg}{\mathrm{deg}}
\newcommand{\km}{\mathrm{km}}
\newcommand{\cm}{\mathrm{cm}}
\newcommand{\mm}{\mathrm{mm}}
\newcommand{\ft}{\mathrm{ft}}
\newcommand{\mi}{\mathrm{mi}}
\newcommand{\AU}{\mathrm{AU}}
\newcommand{\ns}{\mathrm{ns}}
\newcommand{\h}{\mathrm{h}}
\newcommand{\yr}{\mathrm{yr}}
\newcommand{\N}{\mathrm{N}}
\newcommand{\J}{\mathrm{J}}
\newcommand{\eV}{\mathrm{eV}}
\newcommand{\W}{\mathrm{W}}
\newcommand{\Pa}{\mathrm{Pa}}

% derived units
\newcommand{\mps}{\m\,\s^{-1}}
\newcommand{\mph}{\mi\,\h^{-1}}
\newcommand{\mpss}{\m\,\s^{-2}}

% random stuff
\sloppy\sloppypar\raggedbottom\frenchspacing\thispagestyle{empty}

\begin{document}

\section*{NYU Physics I---potential energy}

For a conservative force, the potential energy is the line integral of
the force and the force is the derivative of the potential.  This is
the fundamental theorem of calculus but also very fundamental in
physics.  Here we do a non-trivial example.

\paragraph{\theproblem}\refstepcounter{problem}%
Consider a particle subject to a potential energy of the form
\begin{equation}
U = \frac{B}{x^2} - \frac{A}{x} \quad ,
\label{eq:hard}
\end{equation}
where $A$ and $B$ are positive constants.  If $x$ is measured in $\m$,
what are the units of $A$ and $B$?

Consider another particle subject to a potential energy of the form
\begin{equation}
U = \frac{1}{2}\,k\,(x-x_0)^2
\label{eq:easy}
\end{equation}
where $k$ is a constant.  If $x$ is measured in $\m$, what are
the units of $k$?

\paragraph{\theproblem}\refstepcounter{problem}%
Sketch both potentials $U$ vs $x$.

\paragraph{\theproblem}\refstepcounter{problem}%
Take derivatives for each potential, and use these to get the ``force law''
for each potential. Plot the force as a function of position $x$.

\paragraph{\theproblem}\refstepcounter{problem}%
Find the equilibrium position (or positions, if there are more than
one) $x_\mathrm{eq}$ for each of the potentials.  That is, compute the
$x$ positions where there is no force and the potential is at a
minimum.

\paragraph{\theproblem}\refstepcounter{problem}%
Take a second derivative for each potential at the equilibrium point.
What are the units of this second derivative?  Interpret it physically
in terms of force and distance. What does this have to do with the
Taylor series? Discuss.

\paragraph{\theproblem}\refstepcounter{problem}%
In the second potential (\ref{eq:easy}), you know what the dynamics
are: what happens to the system as a function of time if displaced
from equilibrium?  How does this relate to what happens in the first
potential (\ref{eq:hard})?  There is a limit in which these look
\emph{very similar}; what is it?

\paragraph{\theproblem}\refstepcounter{problem}%
Of what kind of system might potential (\ref{eq:hard}) be a model?

\end{document}
