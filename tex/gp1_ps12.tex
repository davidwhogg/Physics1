\documentclass[12pt]{article}
\newcommand{\m}{\mathrm{m}}
\newcommand{\cm}{\mathrm{cm}}
\newcommand{\s}{\mathrm{s}}
\renewcommand{\min}{\mathrm{min}}
\renewcommand{\l}{\ell}
\newcommand{\C}{\mathrm{deg\,C}}
\newcounter{problem}
\begin{document}\thispagestyle{empty}

\section*{NYU General Physics 1---Problem set 12}

\paragraph{Problem~\theproblem:}\refstepcounter{problem}%
\textsl{(a)}~Look up the pressure corresponding to one atmosphere in
kPa.  Starting at sea level, under what
depth $h$ of water will the pressure be 2 atmospheres?  That is, what
depth leads to a pressure increase of one atmosphere?

\textsl{(b)} Same but for mercury.  What is the conversion between mm
of Hg and kPa?

\textsl{(c)} Imagine that the air was incompressible (that is,
constant density).  What height of atmosphere corresponds to one
atmosphere of pressure?  Use the STP density for air.

\textsl{(d)} The air is \emph{not} incompressible, so what really is
the meaning of the height computed in part~\textsl{(c)} (no, the
answer is not ``no meaning at all'')?  Can you think of any industrial
activity on Earth that makes use of the altitude or height computed
in part~\textsl{(c)}?

\paragraph{Problem~\theproblem:}\refstepcounter{problem}%
A cube of ice $2\,\cm$ on a side floats in a glass of water.  The
water is at $0\,\C$ and the glass is in a room at STP.  Look up or
assume what you need in order to solve these problems.

\textsl{(a)} What is the pressure \emph{difference} $\Delta P$ between
the atmospheric pressure and the pressure at the bottom surface of the
ice cube?

\textsl{(b)} If the ice cube melts, does the water level go up or
down?  Why?

\textsl{(c)} If you submerge the ice cube, and hold it at rest under
water, there is a pressure difference $\Delta P$ between the top and
bottom face of the cube.  Is this larger than, smaller than, or the
same as the pressure difference you calculated in part \textsl{(a)}?

\textsl{(d)} If you release the submerged cube, it will accelerate
upwards.  Immediately after release, what is the net force $F$ on the
ice cube?  Can you think of two different ways of calculating this?

% this problem should (a) mention that weight is normal force, (b) replace the buoyant force on the blimp with the total net lift, and (c) say what is ``held fixed'' as the blimp rises in altitude or temperature.

\paragraph{Problem~\theproblem:}\refstepcounter{problem}%
Everything submerged in the Earth's atmosphere is subject to a buoyant
force from the air.  In the following, use a sensible (reasonably
accurate) measure of the density of air at STP.

\textsl{(a)} When you measure your weight on a standard bathroom
scale, you are measuring the \emph{normal force} between yourself and
the floor.  This normal force opposes the \emph{combination} of
gravity and buoyancy.  What is the correction to your weight coming
from buoyancy, roughly?  Express it as a \emph{fraction} of the
gravitational force.  Is this correction positive or negative---that
is, does it increase or decrease the weight measured by the scale?

\textsl{(b)} Look up the ``volume'' of the Goodyear blimp model GZ-22.
Imagine that it is floating in an air atmosphere at STP, and that the
gas inside the blimp is \emph{also} at STP.  What is the approximate
buoyant force on the blimp if it is filled with helium?  What about if
it were filled with hydrogen (molecular hydrogen)?  Compare these
numbers with the gross weight and capacity of the blimp.

\textsl{(c)} Will the buoyant force increase, decrease, or stay the
same as you change the altitude (atmospheric density) or temperature?
Assume that the blimp contents are always at the same pressure as the
exterior air.

\end{document}
