\documentclass[12pt]{article}
\usepackage{url, graphicx, epstopdf}

% page layout
\setlength{\topmargin}{-0.25in}
\setlength{\textheight}{9.5in}
\setlength{\headheight}{0in}
\setlength{\headsep}{0in}
\setlength{\parindent}{1.1\baselineskip}

% problem formatting
\newcommand{\problemname}{Problem}
\newcounter{problem}

% words
\newcommand{\foreign}[1]{\textsl{#1}}
\newcommand{\vs}{\foreign{vs}}

% math
\newcommand{\dd}{\mathrm{d}}
\newcommand{\e}{\mathrm{e}}

% primary units
\newcommand{\rad}{\mathrm{rad}}
\newcommand{\kg}{\mathrm{kg}}
\newcommand{\m}{\mathrm{m}}
\newcommand{\s}{\mathrm{s}}

% secondary units
\renewcommand{\deg}{\mathrm{deg}}
\newcommand{\km}{\mathrm{km}}
\newcommand{\cm}{\mathrm{cm}}
\newcommand{\mm}{\mathrm{mm}}
\newcommand{\ft}{\mathrm{ft}}
\newcommand{\mi}{\mathrm{mi}}
\newcommand{\AU}{\mathrm{AU}}
\newcommand{\ns}{\mathrm{ns}}
\newcommand{\h}{\mathrm{h}}
\newcommand{\yr}{\mathrm{yr}}
\newcommand{\N}{\mathrm{N}}
\newcommand{\J}{\mathrm{J}}
\newcommand{\eV}{\mathrm{eV}}
\newcommand{\W}{\mathrm{W}}
\newcommand{\Pa}{\mathrm{Pa}}

% derived units
\newcommand{\mps}{\m\,\s^{-1}}
\newcommand{\mph}{\mi\,\h^{-1}}
\newcommand{\mpss}{\m\,\s^{-2}}

% random stuff
\sloppy\sloppypar\raggedbottom\frenchspacing\thispagestyle{empty}

\begin{document}

\section*{NYU Physics I---small variations}

\paragraph{\theproblem}\refstepcounter{problem}%
You have a cubic block of ice, $1.000\,\m$ on a side.  By what
fractional amount does its mass decrease if you shave a millimeter off
of it in each dimension (ie, so that it becomes a cubic block of ice,
$0.999\,\m$ on a side)?

\paragraph{\theproblem}\refstepcounter{problem}%
Explain why your answer to the above was so close to $3\times
10^{-3}$.  You have two explanations to give, \textsl{(a)}~one from
the point of view of the three operations you did (shave one face,
then the next, then the next), and \textsl{(b)}~one from the point of
view of small variations (write the formula for the volume $V$ in
terms of the side length $\ell$ and differentiate with respect to
$\ell$).

\paragraph{\theproblem}\refstepcounter{problem}%
Give a general argument---using calculus---that
\begin{equation}
\lim_{\epsilon\rightarrow 0}\,(1+\epsilon)^n=1+n\,\epsilon \quad .
\end{equation}

\paragraph{\theproblem}\refstepcounter{problem}%
With a calculator, compute the sine of the angle $\theta=0.1\,\rad$.
Now compute the error you make if you use $\sin\theta=\theta$.  Now
compare that error to the quantity $\theta^2/2$ and $\theta^3/6$.  Is
it close to either?

\paragraph{\theproblem}\refstepcounter{problem}%
Now compute the error you make if you use
\begin{equation}
\lim_{\theta\rightarrow 0}\,\sin\theta=\theta-\frac{1}{6}\,\theta^3 \quad .
\label{eq:third}
\end{equation}
Do you think the \emph{next} term in the Taylor series for sine will
be negative or positive on this basis?

\paragraph{\theproblem}\refstepcounter{problem}%
Inside of what angle is the approximation $\sin\theta=\theta$ good to
one percent?  What is this angle in degrees?

\paragraph{\theproblem}\refstepcounter{problem}%
Now square the expression in Equation~(\ref{eq:third}) and compare it
to the second-order expression for cosine:
\begin{equation}
\lim_{\theta\rightarrow 0}\,\cos\theta=1-\frac{1}{2}\,\theta^2 \quad .
\end{equation}
Show that this is consistent with the trigonometric identity
\begin{equation}
\sin^2\theta = \frac{1 - \cos(2\,\theta)}{2} \quad .
\end{equation}

\end{document}
