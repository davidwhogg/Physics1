\documentclass[12pt]{article}
\begin{document}
\thispagestyle{empty}

\section*{NYU General Physics 1---Estimation problems}

\paragraph{Problem 1:}

Previously this semester, you have computed the power expended by your
car traveling at $55~\mathrm{mi\,h^{-1}}$ by (a) considering fuel
consumption, and, indirectly, by (b) estimating air drag on the moving
car.  Reconcile your estimates; \textit{i.e.,} figure out how they are
related and check that the relation is satisfied, at least to order of
magnitude.

\paragraph{Problem 2:}

How many piano tuners are there in the City of New York?  Estimate
it, and then check your answer with the phone book or WWW.

\paragraph{Problem 3:}

You are stuck in Kansas and you see an airplane flying overhead
between LA and NY.  Estimate the angular speed (radians per second) at
which the plane passes across your field of view.

\paragraph{Problem 4:}

Estimate the mass of a fully loaded commercial airliner, before
take-off.  Don't forget the fuel.  Estimate the minimum total thrust
(in Newtons) of the engines if they can make the plane ascend at an
angle of 20~deg to the horizontal.

\paragraph{Problem 5:}

An airplane has to bank to turn.  What is the radius of curvature of
the trajecrory of a commercial airliner at cruising speed banked at
35~deg?

\paragraph{Problem 6:}

When a bird lands on a power line, it sends a transverse wave down the
line.  Estimate the speed of the wave.  Use things you know about
power lines to estimate the linear mass density of the line and the
tension (remember a problem set you did?).

\paragraph{Problem 7:}

What is the approximate moment of inertia $I$ of a figure skater
spinning with her hands outstretched.  If her period of rotation is
1.5~s, what is her total kinetic energy of rotation?

\paragraph{Problem 8:}

When your car is rolling down the street, what fraction of its kinetic
energy is in the form of translation of the total mass, and what
fraction is in the form of rotation of the wheels?

\paragraph{Problem 9:}

Compare the force of air resistance to the force of gravity acting on
a baseball that has been hit hard enough to fly 300~m if there were no
air resistance.  For the ``formula'', recall a worksheet you did?  Is
the assumption of no air resistance a good assumption or bad one?

\paragraph{Problem 10:}

You walk at a pace that is set, in part, by the natural frequency of
oscillation of your leg, treated as a pendulum.  Estimate this
frequency first by treating your leg as a massless rod with a mass on
the end, and then, more realistically, as a massive rod, pivoted at
one end.  Does your answer seem reasonable?

\paragraph{Problem 11:}

What is the most valuable ingredient of Thanksgiving dinner, in terms
of dollars per kg?  What is its value in dollars per kg?

\paragraph{Problem 12:}

What is the total mass of the students in this class?

\paragraph{Problem 13:}

What is the kinetic energy $K$ of a typical car going at freeway
speeds?  What is the typical power output $P$ of a typical car engine?
What are the dimensions of the ratio $K/P$, and what does that ratio
``mean''?

\paragraph{Problem 14:}

What is the typical thermal speed of a molecule in the air?  Recall
that $k\,T$ is roughly its kinetic energy.  What else do you have to
assume?  Compare your answer to the sound speed.

\paragraph{Problem 15:}

When you boil water it expands in volume by a factor of (very roughly)
1000.  How much work does it take to expand 1 gram of water by this
factor, assuming the expansion happens at atmospheric pressure?  How
does your answer compare to the latent heat of vaporization?  Do you
think these two energies ought to be related in any way?

\end{document}
