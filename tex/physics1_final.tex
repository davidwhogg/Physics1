\documentclass[12pt, letterpaper]{article}
\usepackage{url, graphicx, epstopdf}

% page layout
\setlength{\topmargin}{-0.25in}
\setlength{\textheight}{9.5in}
\setlength{\headheight}{0in}
\setlength{\headsep}{0in}
\setlength{\parindent}{1.1\baselineskip}

% problem formatting
\newcommand{\problemname}{Problem}
\newcounter{problem}

% words
\newcommand{\foreign}[1]{\textsl{#1}}
\newcommand{\vs}{\foreign{vs}}

% math
\newcommand{\dd}{\mathrm{d}}
\newcommand{\e}{\mathrm{e}}

% primary units
\newcommand{\rad}{\mathrm{rad}}
\newcommand{\kg}{\mathrm{kg}}
\newcommand{\m}{\mathrm{m}}
\newcommand{\s}{\mathrm{s}}

% secondary units
\renewcommand{\deg}{\mathrm{deg}}
\newcommand{\km}{\mathrm{km}}
\newcommand{\cm}{\mathrm{cm}}
\newcommand{\mm}{\mathrm{mm}}
\newcommand{\ft}{\mathrm{ft}}
\newcommand{\mi}{\mathrm{mi}}
\newcommand{\AU}{\mathrm{AU}}
\newcommand{\ns}{\mathrm{ns}}
\newcommand{\h}{\mathrm{h}}
\newcommand{\yr}{\mathrm{yr}}
\newcommand{\N}{\mathrm{N}}
\newcommand{\J}{\mathrm{J}}
\newcommand{\eV}{\mathrm{eV}}
\newcommand{\W}{\mathrm{W}}
\newcommand{\Pa}{\mathrm{Pa}}

% derived units
\newcommand{\mps}{\m\,\s^{-1}}
\newcommand{\mph}{\mi\,\h^{-1}}
\newcommand{\mpss}{\m\,\s^{-2}}

% random stuff
\sloppy\sloppypar\raggedbottom\frenchspacing\thispagestyle{empty}

\pagestyle{plain}

\begin{document}

\examheader{Final Exam}

\begin{problem} (From Term Exam 1)
  If the radius of the Moon is about $1/4$ that of the Earth,
  and the Moon and Earth are made of similar things,
  what is the mass of the Moon?
  Give your answer as a number in kg. No need to be precise!
\end{problem}

\vfill

\begin{problem} (From Term Exam 2)
  A block of mass $M$ sits on an inclined plane, inclined at $\theta=20$\,deg to the horizontal.
  There is a coeffient of friction of $\mu = 0.9$ between the block and the plane
  (both static and kinetic friction coefficients are 0.9).
  What is the magnitude of the frictional force on the block?
  Give your answer in terms of any combination of the symbols $M$, $g$, $\theta$, and $\mu$.
\end{problem}

\vfill

\begin{problem} (From Term Exam 3)
  A block of mass $4\,M$ is moving at speed $u$ in the $+\hat{x}$ direction,
  and a block of mass $M$ is moving at speed $u$ in the $-\hat{x}$ direction.
  What is the velocity of the center of mass of this two-block system?
\end{problem}

\vfill
~\clearpage

\begin{problem} (From Term Exam 4)
  An ice cube floats in a glass of water.
  When the ice cube melts, does the water level go up or down or stay the same?
  Explain why.
\end{problem}

\vfill

\begin{problem} (From Term Exam 5)
  The Earth orbits, at a mean distance of $1\,\au$, the Sun.
  Imagine instead that the Earth orbits, at the same mean distance, a black hole that is 36 times the the mass of the Sun.
  What is the length of the year in this new situation?
\end{problem}

\vfill

\begin{problem} (From Term Exam 6)
  A thin hoop of radius $R=0.05\,\m$ and mass $m=0.1\,\kg$ rolls without slipping down an inclined plane.
  A solid ball of radius $R=0.02\,\m$ and mass $m=0.05\,\kg$ rolls without slipping down that same plane.
  Which one has the larger acceleration $|\vec{a}|$, and \emph{why?}
\end{problem}

\vfill
~\clearpage

\begin{problem} (From Problem Set 1)
  Imagine you have a mass $M$, a length $h$, a velocity $v$, and an
  acceleration $g$. Give two qualitatively different combinations of these
  that will have units of force.
\end{problem}

\vfill

\begin{problem} (From Problem Set 2)
  What is the mass density of diatomic nitrogen ($\mathrm{N}_2$) gas at standard
  temperature and pressure? Give a numerical value in SI units.
\end{problem}

\vfill

\begin{problem} (From Problem Set 3)
  In a 200-m race, the winner crosses the halfway mark (ie, 100~m) at
  $t_{1/2}=10.12$~s and the finish line at $t_{\rm f}=19.32$~s.
  Make a kinematic model of the runner's behavior as (i)~constant
  acceleration $a$ from rest for the period $0<t<t_a$ followed by
  (ii)~constant speed $v=a\,t_a$ during the period $t_a<t<t_{\rm f}$.
  Is $t_a$ less than or greater than $t_{1/2}$?
\end{problem}

\vfill
~\clearpage

\begin{problem} (From Problem Set 4)\marginpar{\includegraphics[width=\marginparwidth]{../mp/tackle_blocks.pdf}}
  Here is a non-trivial machine that delivers a kind of mechanical
  advantage. To what mass can I set $m_2$ such that the system would
  be perfectly ``balanced''; that is, for there to be no
  net acceleration of either block?
  Assume the usual massless strings and frictionless pulleys.
\end{problem}

\vfill

\begin{problem} (From Problem Set 5)
  A machine at a packaging facility places stationary packages of mass
  $m$ onto a horizontal conveyor belt that is moving packages steadily
  and horizontally at speed $v$. Once placed on the belt, the packages
  start moving at speed $v$; that is, they are rapidly accelerated.
  What is the momentum change $|\Delta \vec{p}|$ for each package as it
  gets placed on to the belt, and what is the kinetic energy change $\Delta K$?
\end{problem}

\vfill

\begin{problem} (From Problem Set 7)
  What is the length of a classical pendulum (light string, heavy point
  mass at the end) that oscillates in small oscillations with a roughly
  6.3-s period? use $g=10\,\mpss$, and give a numerical answer in m.
\end{problem}

\vfill
~\clearpage

\begin{problem} (From Problem Set 8)
  A very thin ladder of length $L$ and mass $M$ leans against a vertical
  wall, on a horizontal floor, making an angle of $\theta$ with respect
  to the wall.  Imagine that there is a large coefficient of friction
  $\mu$ at the floor so that the ladder is in static
  equilibrium, but assume that the wall is effectively frictionless.
  Draw a free-body diagram for the ladder, showing all
  forces acting.
\end{problem}

\vfill

\begin{problem} (From Problem Set 9)
  What is the pressure gradient in air at sea level at STP?
  What is the pressure gradient in water?
  Give numerical answers with units of $\Pa\,\m^{-1}$.
\end{problem}

\vfill

\begin{problem} (From Problem Set 10)
  A figure skater spins in place on frictionless ice at
  angular speed $\omega_i$ with her hands outstretched.  She has a total
  moment of inertia $I_i$.  As the skater draws her hands into her body,
  her moment of inertia decreases to $I_f=I_i/2$. As she does this, does her kinetic
  energy $K$ increase, decrease, or stay the same?
\end{problem}

\vfill
~\clearpage

\begin{problem} (From Problem Set 11)
  What is the orbital time of the International Space Station?
\end{problem}

\vfill

\begin{problem} (From Problem Set 12)
  Sketch orbits of fixed semi-major axis but increasing
  eccentricity, from a circular orbit to one that is close to radial
  (eccentricity close to unity). Draw all your orbits on one diagram, all orbiting the same
  massive object. Make sure you show the location of this object.
\end{problem}

\vfill

\begin{problem} (From Problem Set 13)
  What is $\gamma$ to first order in $\beta^2$ for $\beta<< 1$?
  That is, construct a Taylor Series for $\gamma$ in terms of
  $\beta^2$ and give the zeroth-order term (1) and then the first-order
  term.
\end{problem}

\vfill
~\clearpage

\begin{problem} (From Problem Set 14)
  A particle of rest mass $M$ is at rest.
  It decays into two photons, which move in the positive and negative $x$ directions.
  What are their two energies?
\end{problem}

\vfill

\begin{problem} (From the Recitation on bouncing)
  Consider a pool shot to be an elastic collision between balls of equal
  masses. The cue ball is moving before the collision and the object
  ball is at rest. Use a precise geometric argument to explain why,
  immediately after the collision, the
  object ball and the cue ball move along paths that are at right angles to one another.
\end{problem}

\vfill

\begin{problem} (From the Recitation on oscillations)
  Consider a mass $M$ on a spring of spring constant $k$, released from
  rest at $t=0$ but from a distance $X$ (in the $x$-direction, which is
  parallel to the spring) away from the equilibrium position for the
  mass.  Assume there are no other forces acting.
  The mass will oscillate.
  Plot three curves---the kinetic energy, the potential energy, and the total
  mechanical energy---of the mass as a function of time, all on the same plot,
  for a few periods of oscillation.
  Be sure to label all three curves clearly.
\end{problem}

\vfill
~\clearpage

\begin{problem} (From the Recitation on the ideal gas law)
  Imagine you have $N$ particles, each of mass $m$, in a cubical box of volume $V$.
  Assume that each particle bounces elastically off of the walls.
  Imagine that each particle has a different velocity, but there is some
  mean squared velocity $\langle v_x^2\rangle$ in the $x$-direction. What is the
  mean pressure caused by the particle collisions
  with one of the two walls that is oriented normal to the $x$ direction?
\end{problem}

\vfill

\begin{problem} (From the Recitation on spacetime intervals)
  Draw a spacetime diagram showing the four events
  $A=(c\,t_A,x_A)=(0\,\m,0\,\m)$, $B=(1\,\m,1\,\m)$, $C=(1\,\m,0\,\m)$,
  and $D=(0\,\m,1\,\m)$.
  Then draw another spacetime diagram showing these same 4 events
  after a Lorentz transformation has been applied with $\beta=(3/5)$ in the x-direction.
  That is, draw here the two diagrams you drew in Recitation.
\end{problem}

\vfill

\begin{problem} (From the Recitation on spacetime diagrams)
  Imagine there is a galaxy flying away from you at velocity $v$.
  When the galaxy is moving away, it sends back to you a light
  signal every $T'=3.3\,\ns$ (as recorded in the galaxy's rest frame).
  What is the time interval between the arrival events (arrivals of
  the signals from the galaxy) according to you (that is, in your
  frame). Give your answer in terms of $T'$ and $\beta$ (where $\beta=v/c$).
\end{problem}

\vfill
~\clearpage

\begin{problem} (From all the Lectures)
  What thing did you learn in the Lectures that you thought was the best or most impressive thing about physics?
\end{problem}

\vfill

\begin{problem} (From all the Lectures)
  What thing did you learn in the Lectures that you thought was the most disturbing or unsettling thing about physics?
\end{problem}

\vfill
~\clearpage

~
\vfill
\textsl{This page intentionally left blank.}
\vfill
~
\end{document}
