\documentclass[12pt]{article}
\usepackage{url, graphicx, epstopdf}

% page layout
\setlength{\topmargin}{-0.25in}
\setlength{\textheight}{9.5in}
\setlength{\headheight}{0in}
\setlength{\headsep}{0in}
\setlength{\parindent}{1.1\baselineskip}

% problem formatting
\newcommand{\problemname}{Problem}
\newcounter{problem}

% words
\newcommand{\foreign}[1]{\textsl{#1}}
\newcommand{\vs}{\foreign{vs}}

% math
\newcommand{\dd}{\mathrm{d}}
\newcommand{\e}{\mathrm{e}}

% primary units
\newcommand{\rad}{\mathrm{rad}}
\newcommand{\kg}{\mathrm{kg}}
\newcommand{\m}{\mathrm{m}}
\newcommand{\s}{\mathrm{s}}

% secondary units
\renewcommand{\deg}{\mathrm{deg}}
\newcommand{\km}{\mathrm{km}}
\newcommand{\cm}{\mathrm{cm}}
\newcommand{\mm}{\mathrm{mm}}
\newcommand{\ft}{\mathrm{ft}}
\newcommand{\mi}{\mathrm{mi}}
\newcommand{\AU}{\mathrm{AU}}
\newcommand{\ns}{\mathrm{ns}}
\newcommand{\h}{\mathrm{h}}
\newcommand{\yr}{\mathrm{yr}}
\newcommand{\N}{\mathrm{N}}
\newcommand{\J}{\mathrm{J}}
\newcommand{\eV}{\mathrm{eV}}
\newcommand{\W}{\mathrm{W}}
\newcommand{\Pa}{\mathrm{Pa}}

% derived units
\newcommand{\mps}{\m\,\s^{-1}}
\newcommand{\mph}{\mi\,\h^{-1}}
\newcommand{\mpss}{\m\,\s^{-2}}

% random stuff
\sloppy\sloppypar\raggedbottom\frenchspacing\thispagestyle{empty}

\begin{document}

\section*{NYU Physics I---Problem Set 2}

Due Tuesday 2025 September 16 by 23:59 (NYC time) on Brightspace. Make sure
your submission is all in one file (one pdf, say) so that it is easy for
the grader to manage. Also, make sure that your handed in work is your
own work, and gives credit to all people and all sources who influenced
your answers and results.

\paragraph{Problem~\theproblem:}\refstepcounter{problem}%
One of the top-fuel drag-racing records is held by Brittany Force, who has a
time of 3.645\,s on a straight track that is 1000\,ft from start to finish.
In drag racing, the car starts at rest, and accelerates for the full time.
Under the approximation that the acceleration of the car is constant, compute

\textsl{(a)} What is her acceleration? Give your answer in $\m\,\s^{-1}$.
Also give the answer in terms of $g$, the magnitude of the acceleration due
to gravity at the Earth's surface. That is, how many ``gees'' did she pull?
Be careful with all your units!

\textsl{(b)} At what time do you compute that she should have passed the halfway point (500\,ft)?

\textsl{(c)} How fast should she have been going as she crossed the finish line?
For this part, convert your answer back to mph (the units drag racing uses).

\textsl{(d)} Now look up the speed she actually \emph{was} going as she crossed
the finish line in this record-breaking run on 2025-07-25.
If the speed you find is different from your answer, why do you think that is?

\paragraph{Problem~\theproblem:}\refstepcounter{problem}%
\textsl{(a)} Look up the masses and sizes of neutral atoms---or a
scaling law for the sizes of the outer electron shell for neutral
atoms---and compute the densities (in $\kg\,\m^{-3}$) of a Hydrogen
atom, an Oxygen atom, an Iron atom, and a Uranium atom, in each case
assuming that the atom is spherical. Compare the
last two to the densities of iron and uranium in metallic form.

\textsl{(b)} Compute (or look up!), using the ideal gas law, the mass
density of diatomic Nitrogen gas at ``standard temperature and
pressure''. How close do you think this estimate will be to the
density of air at STP? Like will it be off by a factor of 2, by 10
percent, by 1 percent, by 0.1 percent, or what? Explain what leads
to the discrepancy.

\paragraph{Problem~\theproblem:}\refstepcounter{problem}%
Re-do the numerical integration worksheet problem from recitation this
week (download it from the course web site if you have forgotten), but
this time do it with a computer spreadsheet and use a time resolution
($\Delta t$) of $0.01\,\s$.  No need to hand in the whole spreadsheet,
or the answers to all the questions, but hand in a graph (plotted by
your spreadsheet program) of the position as a function of time for
the duration $0<t<2\,\s$.  Make sure your axes are clearly labeled and
``calibrated'' in units of m and s.

\paragraph{Problem~\theproblem}\refstepcounter{problem}%
Now do something similar to the above, but analytically.  Consider a
stone thrown at $t=0$ precisely upwards (in the $y$ direction, for
definiteness) at $1.5\,\mps$, with an initial position (launch point)
at $y=0$.  Ignore air resistance!  Make very careful plots of the
vertical position $y$, the vertical velocity $v_y$, and the vertical
acceleration $a_y$ of the stone as a function of time for the duration
$0<t<0.4\,\s$.  Carefully label the time and $y$-position of
the peak of the trajectory (the highest point) in all three curves,
and the time at which the trajectory passes back through $y=0$, if
that ever happens. Be very careful to include units with all of your
numbers and labels!

\paragraph{Problem~\theproblem}\refstepcounter{problem}%
When an airplane turns a corner, it banks (tilts). When the plane is
flown correctly, the tangent (yes, the trig function ``tan'') of this
tilt angle is set by the ratio of the transverse acceleration
(centripetal acceleration) to the acceleration due to gravity.

\textsl{(a)} If you see a commercial jet aircraft tilted at $30\,\deg$
because it is turning, about what do you think the radius of the turn
is? Clearly state your assumptions! You might have to look up or
estimate the speed at which planes fly. Does your answer seem
reasonable given what you know about planes and travel?

\textsl{(b)} Draw a free-body diagram for the turning airplane,
showing the gravitational force, the lift force from the wings, the
thrust force from the engines, and the drag force from air resistance.
These are the \emph{only} four forces you need to have to explain the
turning airplane. Make sure your diagram (maybe you need to show two
views of it?) clearly shows the force directions in three-dimensional
space.

\end{document}
